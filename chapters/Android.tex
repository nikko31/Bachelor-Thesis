\section{Stile architetturale REST}
Il Web è popolato da articoli e trattazioni relative alle architetture REST che tristemente ripetono le medesime battute. Pertanto, al fine di minimizzare la monotonia, in questo paragrafo ci limiteremo a ricordare brevemente i concetti fondamentali, riducendo al minimo la ripetizione delle storielle ormai usurate. L’acronimo REST, REpresentational State Transfer (“trasferimento dello stato di rappresentazione”) deriva dalla tesi di dottorato di Roy Fielding intitolata “Architectural Styles and the Design of Network-based Software Architectures” (“stili architetturali e progettazione di architetture software basate sul networking”) [2]; concediamoci una battuta ricorrente: per essere una tesi di dottorato, non è poi così difficile da leggere. Fielding è uno dei principali autori di del protocollo HTTP, HypertText Transfer Protocol, (“protocollo di trasferimento degli ipertesti”) versione 1.0 e 1.1.

REST non è un’architettura bensì uno “stile architetturale” formato da vincoli, linee guida e best practice. La formulazione dello stile REST è stata ottenuto dopo un’attenta analisi effettua da Fielding sulle risorse e le tecnologie disponibili per la creazione di applicazioni distribuite. L’assioma di base è che, senza imporre alcun vincolo, i sistemi tendono “naturalmente” a evolvere in maniera entropica, generando le conseguente negative che tutti noi ben conosciamo (sistemi costosi da creare, difficili da mantenere e da far evolvere, etc.).

Con questo assunto in mente, Fielding ha iniziato la sua esplorazione nel dominio degli stili delle architetture distribuite partendo dal limite inferiore da lui definito “spazio nullo” (null space), ossia il “Far West”, rappresentato da organizzazioni dotate di sistemi a basso grado di maturità in cui tutte le risorse tecnologiche sono disponibili, tutti gli stili sono ammessi, senza regole ne’ limiti. Continuando lungo la direttrice evolutiva, Fielding ha poi definito lo stato di totale maturità caratterizzato da sistemi che rispettano le regole da lui definite e che quindi possono essere definiti compatibili con le guideline REST (RESTful). Queste regole sono condensate nei seguenti sei vincoli, di cui i primi cinque sono obbligatori, mentre l’ultimo è facoltativo. Vediamo quindi i 5+1 vincoli cui un sistema deve sottostare per essere definito RESTful.