Mi ritengo soddisfatto del progetto di tesi svolto.
Innanzitutto, sono stati affrontati
problemi pratici che riguardano la produzione di software, sia dal lato progettuale sia da quello implementativo.
L'adozione di frameworks è risultata molto efficacie, rendendo lo sviluppo del software più semplice permettendo la creazione di una struttura  adattabile e flessibile risolvendo in tal modo numerose problematiche.\\
GWT ha permesso di trasformare trasparentemente e automaticamente il codice prodotto in una RIA Ajax con tanto di client che lavora in modalità asincrona direttamente dentro il browser senza preoccuparsi di codice JavaScript.
Inoltre, grazie al pattern design adottato, abbiamo potuto lavorare efficacemente in team. \\
Hibernate ha permesso di ridurre significativamente i tempi  di sviluppo altrimenti
impiegati in attività manuali di gestione dei dati in SQL e JDBC risolvendo problemi legati all'impedence mismatch e riducendo significativamente la quantità di
codice per la persistenza rendolo altamente manutenibile.
