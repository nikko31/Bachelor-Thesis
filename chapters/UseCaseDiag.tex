\subsection*{Use Case Diagram}
Lo Use Case Diagram è una descrizione di un comportamento particolare di un sistema dal punto di vista dell'utente. Per gli sviluppatori, gli use case diagrams rappresentano uno strumento notevole: tramite tali diagramma, essi possono agevolmente ottenere una idea chiara dei requisiti del sistema dal punto di vista utente e quindi scrivere il codice senza timore di non aver recepito bene lo scopo finale.

Dal documento SRS si ricava il diagramma sottostante (figura\ref{fig:UseCDiagramFirst}) che mette in evidenza le funzioni principali accessibili dall'utente.
L'utente infatti, accedendo alla webapp, potrà visualizzare la mappa di un determinato piano in tre differenti modalità ( RF04, RF05, RF06).
Esso potrà anche cercare le aule occupate da una determinata persona inserendo il suo nome in una specifica barra di ricerca (RF07).
Dalla mappa si può selezionare un'aula e ottenere le informazioni relative ad essa (RF08) e successivamente modificare tali informazioni.

\begin{figure}[!htb]
\centering%
\includegraphics[scale=0.5]{UseCFirst.png}%
\caption{Use Case Diagram delle funzioni principali che l'utente può svolgere.}\label{fig:UseCDiagramFirst}%
\end{figure}


