La grafica vettoriale \`e una tecnica utilizzata in computer grafica per descrivere 
un'immagine. 
Un' immagine \`e descritta mediante un insieme di primitive geometriche che definiscono punti, linee, curve e poligoni ai quali possono essere attribuiti colori e anche sfumature. La grafica vettoriale presenta  vantaggi rispettio alla grafica raster\footnote{La grafica raster, o bitmap, \`e una tecnica per descrivere un'immagine digitale. \`E radicalmente diversa rispetto alla grafica vettoriale in quanto l’immagine \`e composta da una griglia di punti detti pixel, di forma quadrata.}; i principali vantaggi sono:
\begin{itemize}
\item possibilit\`a di esprimere i dati in una forma direttamente comprensibile ad un essere umano;
\itempossibilit\`a di esprimere i dati in un formato che occupi (molto) meno spazio rispetto all'equivalente raster;
\item possibilit\`a di ingrandire l'immagine arbitrariamente, senza che si verifichi una perdita di risoluzione dell'immagine stessa.
\end{itemize}