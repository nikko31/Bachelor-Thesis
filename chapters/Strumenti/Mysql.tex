MySQL è un Relational database management system (RDBMS), composto da un client con interfaccia a caratteri e un server, entrambi disponibili sia per sistemi Unix che per Windows, anche se prevale un suo utilizzo in ambito Unix.

Dal 1996 supporta la maggior parte della sintassi SQL e si prevede in futuro il pieno rispetto dello standard ANSI. 
MySQL e' il DBMS relazionale Open Source piu' diffuso al mondo (Questo è lo slogan ufficiale:  MySQL :The world’s most popular open source database). 
L’aumento esponenziale della diffusione 
di MySQL è spiegabile con l’enorme 
successo del movimento open source.    
I principali punti di forza sono:  
\begin{itemize}
\item Gratis per l'utilizzo come open source.
Il prodotto è open source, quindi 
viene fornito oltre al prodotto anche il 
codice sorgente e questo consente all'utente di vedere come funzionano i programmi e di modificarli a seconda delle proprie esigenze. 
\item Eccezionale diffusione, soprattutto
 per le applicazioni web.  
\item Disponibile anche con una licenza commerciale e supporto tecnico.  
\item Leggero e di poco impatto sui server su cui viene installato . 
\item Semplice nell'utilizzo, nella configurazione e nell'amministrazione
\item Elevate prestazioni. 
\item Consente l'utilizzo di differenti storage engine tra cui scegliere.
\item Disponibile per una grande varietà di piattaforme. 
\\end{itemize}