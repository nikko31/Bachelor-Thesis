Il  linguaggio Java  \`e  un  linguaggio  di  programmazione  orientato  agli  oggetti,  creato  da  James 
Gosling  e  altri  ingegneri  di  Sun  Microsystems.  Il  gruppo  inizi\`o  a  lavorare  nel  1991,  il  linguaggio 
inizialmente si chiamava Oak. Il nome fu successivamente cambiato in Java a causa di un problema di  copyright  (il  linguaggio  di  programmazione  Oak  esisteva  gi\`a nel  1991).  Java  fu  annunciato ufficialmente il 23 maggio 1995 a SunWorld. La piattaforma di programmazione Java \`e fondata sul 
linguaggio  stesso,  sulla  Java  Virtual Machine  (JVM) e  sulle  API.  Java  \`e  un  marchio  registrato  di Sun Microsystems.
\subsection{Panoramica del linguaggio JAVA}
Java venne creato per soddisfare quattro scopi: 
\begin{itemize}
\item essere orientato agli oggetti
\item essere indipendente dalla piattaforma
\item contenere strumenti e librerie per il networking
\item essere progettato per eseguire codice da sorgenti remote in modo sicuro
\end{itemize}
\subsection*{Orientamento agli oggetti}
Java \`e un linguaggio object-oriented. La programmazione orientata agli oggetti (OOP, Object Oriented Programming) \`e un paradigma di programmazione che permette di definire oggetti software in grado di interagire gli uni con gli altri attraverso lo scambio di messaggi.  
La programmazione ad oggetti prevede di raggruppare in una zona circoscritta del codice sorgente (chiamata classe) la dichiarazione delle strutture dati e delle procedure che operano su di esse. Le classi, quindi, costituiscono dei modelli astratti, che a tempo di esecuzione vengono invocate per instanziare o creare oggetti software relativi alla classe invocata. Questi ultimi sono dotati di attributi (dati) e metodi (procedure) secondo quanto definito/dichiarato dalle rispettive classi.

La programmazione ad oggetti \`e particolarmente adatta nei contesti in cui si possono definire delle relazioni di interdipendenza tra i concetti da modellare (contenimento, uso, specializzazione). 

Tra i vantaggi della programmazione orientata agli oggetti abbiamo:
\begin{itemize}
\item fornisce un supporto naturale alla modellazione software degli oggetti del mondo reale o del modello astratto da riprodurre
\item permette una pi\`u facile gestione e manutenzione di progetti di grandi dimensioni
\item permette una pi\`u facile gestione e manutenzione di progetti di grandi dimensioni
\item l'organizzazione del codice sotto forma di classi favorisce la modularit\`a e il riuso di codice
\end{itemize}
\subsection*{Indipendenza dalla piattaforma}
L'indipendenza dalla piattaforma significa che l’esecuzione di programmi scritti  in  Java  deve  avere  un  comportamento  simile 
in contesti di esecuzione diversi. Per raggiungere questo obiettivo il codice Java viene compilato in un linguaggio intermedio bytecode. Il bytecode \`e un insieme di istruzioni altamente ottimizzate eseguibili dalla Java Virtual Machine(JVM), disegnata inizialmente come interprete per tale linguaggio.

La traduzione di un programma Java in bytecode rende molto piu semplice l'esecuzione di un programma in una vasta gamma di ambienti perch\`e solo la JVM deve essere implementata per ogni piattaforma.
\subsection*{Esecuzione sicura del codice remoto}
La piattaforma Java ha caratteristiche progettate per aumentare la sicurezza delle applicazioni Java:
\begin{itemize}
\item \textbf{La JVM:}\\
La JVM esegue una verifica del bytecode prima di eseguirlo per prevenire l'esecuzione di operazioni non sicure e per far rispettare vincoli a runtime. La piattaforma non permette ai programmi di eseguire alcune operazioni insicure e controlli manuali sull'allocazione e deallocazione della memoria.
\item \textbf{Il Security Manager:}\\
La piattaforma fornisce un security manager che permette agli utenti di eseguire codice bytecode inaffidabile in un ambiente \textit{sandbox} progettato per proteggerli da software pericoloso o scritto male evitando che tale codice abbia accesso ad alcune caratteristiche e API della piattaforma.
Il security manager permette di assegnare ai programmi Java una firma digitale in modo tale che l'utente possa scegliere di dare i privilegi a software con firma digitale valida proveniente da entità di fiducia.
\item \textbf{Le API:}\\
Viene fornita una serie di API orientate alla sicurezza, come algoritmi standard di crittografia,autenticazione e protocolli di comunicazione sicuri.
\end{itemize}

\subsection{La piattaforma JAVA}
La  piattaforma  Java  \`e  una  piattaforma  solo software eseguita sopra ad  una  piattaforma hardware di base che pu\`o essere un computer, una tv, un telefono cellulare, una smart card, ecc...
\\ La piattaforma Java \`e composta da due blocchi: 
\begin{itemize}
\item la Java Virtual Machine (JVM)
\item la Java Application Program Interface (API)
\end{itemize} 
La JVM \`e  la  base  della  piattaforma  Java, mentre la Java API \`e  una  collezione  di componenti software pronti all’uso per losvolgimento dei pi\`u disparati compiti.
\subsection*{Java Virtual Machine} 
La JVM consiste in:
\begin{description}
\item [Class loader:]
carica le classi che formano il bytecode, sia dell'applicazione Java, sia delle API Java necessarie per l'esecuzione da parte dell'interprete Java.

\item [Class verifier:]
controlla che il bytecode sia valido, che non superi i limiti superiori o inferiori dello stack, assicura non esegua aritmetica dei puntatori (che potrebbe potenzialmente portare ad una violazione di memoria). Se il bytecode passa tutti questi controlli, pu\`o essere eseguito dall'interprete.

\item [Interprete Java:] 
pu\`o essere di varie forme: pu\`o essere un modulo software che interpreta il bytecode in una sola volta oppure pu\`o fare uso di un compilatore just-in-time (JIT, o Just-In-Time compiler) che traduce a run-time il bytecode in codice nativo della macchina ospitante e lo salva in memoria durante l'esecuzione. \`E anche possibile utilizzare un sistema "misto", in cui il JIT viene applicato solo alle porzioni di codice del programma utilizzate pi\`u frequentemente, mentre il resto viene interpretato.
\end{description}

\subsection*{API Java}
Le API Java raccolgono una gran quantit\`a di componenti disponibili per scrivere applicazioni di qualsiasi genere. Per questo motivo la piattaforma Java \`e disponibile in tre configurazioni a seconda dell'uso che se ne vuole fare:
\begin{itemize}
\item \textbf{Standard Edition.} Fornisce API per le esigenze pi\`u comuni, che permette di scrivere applicazioni stand-alone, applicazioni client e server in un contesto di reti di computer, applicazioni per accesso a database, applicazioni per il calcolo scientifico e di altro tipo.
\item \textbf{Enterprise Edition.} Permette di scrivere applicazioni distribuite.
\item \textbf{Micro Edition.} Permette di scrivere applicazioni per i terminali mobili e, pi\`u in generale, per i dispositivi dotati di poche risorse computazionali (telefoni cellulari, palmari, smart cards ed altri).
\end{itemize}

