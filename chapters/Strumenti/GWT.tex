La necessit\`a di creare un'applicazione Web nasce dai molteplici vantaggi che le applicazioni RIA ( Rich Internet Application ) possiedono nei confronti delle tecnologie alternative. Infatti, rispetto alle applicazioni desktop, non richiedono installazione, gli aggiornamenti sono automatici, sono indipendenti dalla piattaforma utilizzata, pi\`u sicure in quanto girano nel ristretto ambiente del Web browser e maggiormente scalabili perch\`e la maggior parte del lavoro computazionale viene eseguito dal server.\\
Con l'avvento della tecnologia Ajax (Asynchronous JavaScript and XML), lo sviluppo di applicazioni Web si basa su uno scambio di dati in background fra Web browser e server, che
consente l'aggiornamento dinamico di una pagina Web senza esplicito ricaricamento da parte dell'utente. Purtroppo, scrivere applicazioni Ajax \`e molto complicato e perci\`o particolarmente esposto ad errori e bug; questo perch\`e JavaScript \`e un linguaggio piuttosto differente da Java e richiede molta pratica per lo sviluppo; il tutto \`e peggiorato dal fatto che JavaScript tende ad avere differenze in base al browser Web utilizzato, concentrando gli sforzi ed il tempo degli sviluppatori pi\`u sulla parte grafica che sulla logica applicativa. Google Web Toolkit (GWT) nasce proprio per risolvere questi problemi, fornendo un vero e proprio livello di astrazione che nasconde il codice JavaScript e provvede automaticamente ad uniformare le differenze tra i browser. \\
Rilasciato da Google nell'estate 2006 sotto licenza Apache, Google Web Toolkit \`e un set di tool open source che permette agli sviluppatori Web di creare e gestire complesse applicazioni fronted JavaScript scritte in Java. Il codice sorgente Java pu\`o essere compilato su qualsiasi piattaforma con i file Ant inclusi. I punti di forza di GWT sono la riusabilit\`a del codice, la possibilit\`a di realizzare pagine Web dinamiche mediante le chiamate asincrone di Ajax, la gestione delle modifiche, il bookmarking, l'internazionalizzazione e la portabilit\`a fra i differenti browser. 
Nello sviluppo dell'applicativo è stata usata l'ultima release stabile del framework, ovvero laversione 2.7.
\subsection{Architettura GWT 2.7}
GWT \`e caratterizzato da tre componenti base:
\begin{enumerate}
\item Un compilatore Java-JavaScript di alta qualit\`a.
\item La Java Runtime Enviroment (JRE) Emulation library.
\item Interfaccia utente
\end{enumerate}
\subsubsection*{Il compilatore}
Il compilatore Java-Javascript \`e il componente principale di GWT. Si occupa di prendere il codice Java 1.5 e produrre una versione equivalente in JavaScript ed incapsulare la varie differenze tra i browser.
Il compilatore esegue numerosi processi di ottimizzazione del codice, debugging, logging e generazione del codice.
La generazione del codice JavaScript pu\`o essere fatta con uno di questi tre stili:
\begin{description}
\item[-Offuscato]il codice \`e illeggibile, compatto e di piccole dimensioni, quindi \`e raccomandato solamente quando pubblichiamo in produzione la nostra applicazione.
\item[-Formattato]codice leggibile
\item[-Dettagliato]codice ricco di specificatori di package e signature complete: questo stile \`e molto comodo per ricavare informazioni del codice Java dalla console degli errori JavaScript.
\end{description}
\subsubsection*{JRE Emulation library}
Contiene le implementazioni in linguaggio JavaScript delle librerie Java standard maggiormente utilizzate(package java.lang.*,java.sql e java.util.*). Gli altri package(ad esempio java.io.*)sono molto limitati, includendo solo alcune interfaccie. Questa limitazione deriva dal fatto che il codice JavaScript generato da GWT viene eseguito su una sandbox del browser e stampanti.
\subsubsection*{libreria UI}
\`E una libreria User Interface contenente un insieme di interfacce e classi che permettono di disegnare le pagine web (ad es. bottoni, text boxes, immagini, ecc.). Questa \`e la libreria standard principale per creare applicazioni web-based basate su GWT.
\subsection{Comunicazione Client-Server in GWT}
\`E noto che quando si sviluppa una RIA, cosa fondamentale \`e la comunicazione tra il browser(client) e il server.
GWT fornisce differenti strade per comunicare con un server e il formato dei dati da utilizzare dipende dal server con cui si interagisce.
\subsubsection*{Remote Procedure Calls(GWT RPC)}
GWT RPC \`e un framework che permette di facilitare il passaggio di oggetti Java tra client e server (e anche viceversa) attraverso il protocollo HTTP.
E' possibile utilizzare il framework GWT RPC per rendere trasparenti le
chiamate alle servlet Java e lasciare a GWT il compito di prendersi cura dei
dettagli di basso livello come la serializzazione degli oggeti.
Il meccanismo GWT RPC pu\`o esser diviso in tre parti:
\begin{enumerate}
\item Il servizio eseguito sul server
\item Il codice client che invoca il servizio
\item Gli oggetti Java trasmessi tra client e server
\end{enumerate}

\begin{figure}[htbp]
\centering%
\includegraphics[scale=0.7]{RPC1.png}%
\caption{GWT RPC: interazione tra server e client mediante data object}\label{fig:rpc}}%
\end{figure}

Utilizzando GWT RPC tutte le chiamate effettuate dalla pagina HTML al server sono asincrone. Questo significa che le chiamate non bloccano il
client mentre attende una risposta dal server, ma viene eseguito il codice immediatamente successivo.
I vantaggi di effettuare chiamate asincrone rispetto alle pi\`u semplici (per gli
sviluppatori) chiamate sincrone, si riscontrano in una migliore esperienza per gli utenti finali.  innanzitutto, l'interfaccia utente \`e pi\`u reattiva; infatti,
a causa del fatto che nei browser Web il motore JavaScript \`e generalmente di
tipo single-thread, una chiamata sincrona al server genera un "blocco" fino alla conclusione della stessa, rovinando cos\`i l'esperienza dell'utente finale.
Altro vantaggio delle chiamate asincrone \`e che risulta possibile eseguire altri
lavori in attesa della risposta da parte del server; Ultimo vantaggio , ma non meno
importante, \`e che \`e possibile effettuare chiamate multiple al server nello
stesso tempo; tuttavia questo parallelismo risulta fortemente limitato dal
piccolo numero di connessioni che in genere i browser concedono alle singole
applicazioni.\\
I tipi di dato di scambio tra server e client devono essere innanzi tutto serializzabili e possono essere sostanzialmente dei seguenti tipi:
\begin{itemize}
\item Tipi primitivi Java.
\item I wrapper dei tipi primitivi
\item Un sottoinsieme degli oggetti Java Runtime Environemnt (JRE)
\item Qualsiasi tipo definito dall'utente a patto che sia serializzabile\footnote{Le chiamate alle GWT-RPC sono tra codice Javascript e Java e GWT prevede la serializzazione come parte del meccanismo RPC} 
(implementi l'interfaccia Serializable o IsSerializable di GWT)
\end{itemize}

\subsubsection*{Ricevendo dati JSON via HTTP}
Se l'applicazione comunica con un server che non pu\`o ospitare servlet Java, oppure con uno che utilizza gi\`a un'altro formato di dati come JSON o XML, si possono eseguire richieste HTTP per ottenere i dati. GWT fornisce classi HTTP generiche che possono essere utilizzate per fare le richieste, e classi XML e JSON client utilizzabili per processare le risposte.\\
Se si crea un'applicazione che richiede dati da uno o pi\`u web server remoti bisogna evitare le restrizioni SOP(Same Origin Policy)\footnote{Same Origin Policy \`e una misura di sicurezza del browser che limita il codice JavaScript client-side nell'interagire con risorse originate da nomi di dominio, porte e protocolli differenti.}.
