La  piattaforma  Java  \`e  una  piattaforma  solo software che  gira  in  cima  ad  una  piattaforma 
hardware di base che pu\`o essere un computer, una tv, un telefono cellulare, una smart card, 
ecc..  La piattaforma Java \`e composta da due blocchi: 
\begin{itemize}
\item la Java Virtual Machine (JVM)
\item la Java Application Program Interface (API)
\end{itemize} 
La JVM \`e  la  base  della  piattaforma  Java, mentre la Java API \`e  una  collezione  di componenti software pronti all’uso per losvolgimento dei pi\`u disparati compiti.
\subsection*{Java Virtual Machine} 
La JVM consiste di:
\begin{itemize}
\item class loader
\item class verifier
\item l'interprete Java
\end{itemize}
Il class loader carica le classi che formano il bytecode, sia dell'applicazione Java, sia delle API Java necessarie per l'esecuzione da parte dell'interprete Java.

Subito dopo il class verifier controlla che il bytecode sia valido, che non superi i limiti superiori o inferiori dello stack, assicura non esegua aritmetica dei puntatori (che potrebbe potenzialmente portare ad una violazione di memoria). Se il bytecode passa tutti questi controlli, pu\`o essere eseguito dall'interprete.

L'interprete pu\`o essere di varie forme: pu\`o essere un modulo software che interpreta il bytecode in una sola volta oppure pu\`o fare uso di un compilatore just-in-time (JIT, o Just-In-Time compiler) che traduce a runt-time il bytecode in codice nativo della macchina ospitante e lo salva in memoria durante l'esecuzione. \`E anche possibile utilizzare un sistema "misto", in cui il JIT viene applicato solo alle porzioni di codice del programma utilizzate pi\`u frequentemente, mentre il resto viene interpretato.
\subsection*{API Java}
Le API Java raccolgono una gran quantit\`a di componenti disponibili per scrivere applicazioni di qualsiasi genere. Per questo motivo la piattaforma Java \`e disponibile in tre configurazioni a seconda dell'uso che se ne vuole fare:
\begin{itemize}
\item \textbf{Standard Edition.} Fornisce API per le esigenze pi\`u comuni, che permette di scrivere applicazioni stand-alone, applicazioni client e server in un contesto di reti di computer, applicazioni per accesso a database, applicazioni per il calcolo scientifico e di altro tipo.
\item \textbf{Enterprise Edition.} Permette di scrivere applicazioni distribuite.
\item \textbf{Micro Edition.} Permette di scrivere applicazioni per i terminali mobili e, pi\`u in generale, per i dispositivi dotati di poche risorse computazionali (telefoni cellulari, palmari, smart cards ed altri).
\end{itemize}
