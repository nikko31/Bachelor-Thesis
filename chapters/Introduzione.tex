Nello sviluppo tecnico delle applicazioni web e mobile la scelta del framework e della modalità di sviluppo del progetto, che si adatta meglio alle esigenze di mercato del cliente, è un elemento critico che deve essere valutato con grande attenzione visti gli impatti sul business nel medio e breve termine.

Quando si parla di framework si intende un'architettura logica di supporto (spesso un'implementazione logica di un particolare design pattern) su cui un software può essere progettato e realizzato, spesso facilitandone lo sviluppo da parte del programmatore.

Oggi sul mercato esistono diversi tipi di framework, ognuno con caratteristiche e vantaggi differenti. Ciascuno di essi supporta uno o più sistemi operativi, ma non è detto che un framework possa supportare completamente le funzionalità richieste da un’applicazione e da un progetto.

Il lavoro di tesi svolto si pone come obiettivo lo sviluppo in team di un'applicazione web che mette a disposizione degli utenti una visione chiara e completa dell'occupazione delle aule negli spazi universitari. L'applicazione si propone quindi di rendere un servizio sia basato sulla semplice e veloce consultazione delle informazioni riguardanti lo stato delle aule universitarie, sia la manipolazione di tali informazioni.

Oltre all'applicazione web, è stata creata un'applicazione Android per avere un sistema di monitoraggio degli spazi anche su dispositivi mobili.

Per lo sviluppo di questo progetto sono stati utilizzati diversi framework per coprire tutte le richieste e necessità.\\ Nella tesi tratterò esclusivamente la web app, la parte di progetto su cui mi sono soffermato principalmente in fase di sviluppo.
\newpage La tesi è composta da 4 capitoli:
\begin{itemize}
\item Il primo capitolo descrive gli strumenti e tecnologie utilizzate durante la fase di progettazione e sviluppo dell'applicazione;
\item il secondo capitolo espone il progetto stesso e mostra le sue funzionalità nel dettaglio;
\item nel terzo capitolo varrano mostrate le problematiche incontrate e i benefici portati dalle tecnologie usate;
\item infine verranno riportati in appendice i requisiti in forma tabellare e alcune informazioni utili per la manutenzione del sistema.
\end{itemize}