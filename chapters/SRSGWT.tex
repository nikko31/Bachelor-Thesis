\section{Requisiti Fondamentali}
\begin{table}[ht]
\caption{RF: Selezione Edificio}
\begin{center}
\begin{tabular}{|p{3cm}|p{4cm}|p{5cm}|}
\hline
\textbf{RF01}&\textbf{Applicazione Web}&\textbf{Selezione edificio}\\
\hline
\textbf{Input}&\multicolumn{2}{|p{9cm}|}{
Selezione dell'edificio desiderato. }\\
\hline
\textbf{Processo}&\multicolumn{2}{|p{9cm}|}{
L'utente pu\`o selezionare l'edificio desiderato da una ListBox.
}\\
\hline
\textbf{Output}&\multicolumn{2}{|p{9cm}|}{
Nessun output particolare.
}\\
\hline
\end{tabular}
\end{center}
\label{Selezione Edificio}
\end{table}
\begin{table}[ht]
\caption{RF: Selezione piano dell'edificio}
\begin{center}
\begin{tabular}{|p{3cm}|p{4cm}|p{5cm}|}
\hline
\textbf{RF02}&\textbf{Applicazione Web}&\textbf{Selezione piano}\\
\hline
\textbf{Input}&\multicolumn{2}{|p{9cm}|}{
Selezione del piano desiderato. }\\
\hline
\textbf{Processo}&\multicolumn{2}{|p{9cm}|}{
L'utente pu\`o selezionare il piano desiderato di un determinato edificio da una ListBox.
}\\
\hline
\textbf{Output}&\multicolumn{2}{|p{9cm}|}{
Nessun output particolare.
}\\
\hline
\end{tabular}
\end{center}
\label{Selezione Piano}
\end{table}
\begin{table}[ht]
\caption{RF: Selezione Modalit\`a di visualizzazione}
\begin{center}
\begin{tabular}{|p{3cm}|p{4cm}|p{5cm}|}
\hline
\textbf{RF03}&\textbf{Applicazione Web}&\textbf{Selezione Modalit\`a}\\
\hline
\textbf{Input}&\multicolumn{2}{|p{9cm}|}{
Selezione della modalit\`a di visualizzazione desiderata. }\\
\hline
\textbf{Processo}&\multicolumn{2}{|p{9cm}|}{
L'utente pu\`o selezionare una tra le modalit\`a di visualizzazione disponibili da una particolare ListBox.
}\\
\hline
\textbf{Output}&\multicolumn{2}{|p{9cm}|}{
Nessun output particolare.
}\\
\hline
\end{tabular}
\end{center}
\label{Visualizzazione Modalit\`a}
\end{table}
\begin{table}[ht]
\caption{RF: Visualizzazione della mappa del piano selezionato}
\begin{center}
\begin{tabular}{|p{3cm}|p{4cm}|p{5cm}|}
\hline
\textbf{RF04}&\textbf{Applicazione Web}&\textbf{Visualizzazione Mappa}\\
\hline
\textbf{Input}&\multicolumn{2}{|p{9cm}|}{
Premere bottone "Enter" }\\
\hline
\textbf{Processo}&\multicolumn{2}{|p{9cm}|}{
L'utente, dopo aver selezionato l'edificio e il piano desiderati e aver selezionato la modalit\`a "Visualizzazione", cliccher\`a su un determinato bottone per visualizzare la corrispondente mappa.
}\\
\hline
\textbf{Output}&\multicolumn{2}{|p{9cm}|}{
Mappa in modalit\`a visualizzazione.
}\\
\hline
\end{tabular}
\end{center}
\label{Visualizzazione mappa}
\end{table}
\begin{table}[ht]
\caption{RF: Distribuzione Spazi}
\begin{center}
\begin{tabular}{|p{3cm}|p{4cm}|p{5cm}|}
\hline
\textbf{RF05}&\textbf{Applicazione Web}&\textbf{Distribuzione Spazi}\\
\hline
\textbf{Input}&\multicolumn{2}{|p{9cm}|}{
Premere bottone "Enter".}\\
\hline
\textbf{Processo}&\multicolumn{2}{|p{9cm}|}{
L'utente, dopo aver selezionato l'edificio e il piano desiderati e aver selezionato la modalit\`a "Distribuzione Spazi", cliccher\`a su un determinato bottone per visualizzare la corrispondente mappa .
}\\
\hline
\textbf{Output}&\multicolumn{2}{|p{9cm}|}{
mappa che illustra la corrispondente distribuzione degli spazi sfruttando una specifica colorazione delle singole stanze.
}\\
\hline
\end{tabular}
\end{center}
\label{Distribuzione Spazi}
\end{table}
\begin{table}[ht]
\caption{RF: Visualizzazione termine attivit\`a lavorativa}
\begin{center}
\begin{tabular}{|p{3cm}|p{4cm}|p{5cm}|}
\hline
\textbf{RF06}&\textbf{Applicazione Web}&\textbf{Termine Attivit\`a Lavorativa}\\
\hline
\textbf{Input}&\multicolumn{2}{|p{9cm}|}{
Premere bottone "Enter" }\\
\hline
\textbf{Processo}&\multicolumn{2}{|p{9cm}|}{
L'utente, dopo aver selezionato l'edificio e il piano desiderati e aver selezionato la modalit\`a "Termine attivit\`a lavorativa", cliccher\`a su un determinato bottone per visualizzare la corrispondente mappa.
}\\
\hline
\textbf{Output}&\multicolumn{2}{|p{9cm}|}{
Mappa in modalit\`a Termine attivit\`a lavorativa dove \`e possibile consultare le scadenze dei contratti lavorativi delle persone che occupano le aule.
}\\
\hline
\end{tabular}
\end{center}
\label{Visualizzazione termine attivit\`a lavorativa}
\end{table}
\begin{table}[ht]
\caption{RF: Ricerca persone}
\begin{center}
\begin{tabular}{|p{3cm}|p{4cm}|p{5cm}|}
\hline
\textbf{RF07}&\textbf{Applicazione Web}&\textbf{Ricerca Persone}\\
\hline
\textbf{Input}&\multicolumn{2}{|p{9cm}|}{ Scrivere il nome delle persone da cercare e premere bottone "Search". }\\
\hline
\textbf{Processo}&\multicolumn{2}{|p{9cm}|}{
L'utente, dopo aver scritto il nome delle persone che vuole cercare nell'apposita SearchBar, cliccher\`a su un determinato bottone per visualizzare le informazioni riguardanti l'occupazione aule di tali persone.
}\\
\hline
\textbf{Output}&\multicolumn{2}{|p{9cm}|}{
Elenco delle aule occupate dalle persone ricercate
}\\
\hline
\end{tabular}
\end{center}
\label{Ricerca persone}
\end{table}
\begin{table}[ht]
\caption{RF: Visualizzazione informazioni stanza}
\begin{center}
\begin{tabular}{|p{3cm}|p{4cm}|p{5cm}|}
\hline
\textbf{RF08}&\textbf{Applicazione Web}&\textbf{Visualizzazione Informazioni Stanza}\\
\hline
\textbf{Input}&\multicolumn{2}{|p{9cm}|}{Premere su una stanza sulla mappa.}\\
\hline
\textbf{Processo}&\multicolumn{2}{|p{9cm}|}{
L'utente clicca su una stanza della mappa e verranno visualizzate le ralative informazioni.
}\\
\hline
\textbf{Output}&\multicolumn{2}{|p{9cm}|}{
Elenco delle informazioni relative alla stanza e l'elenco delle persone che la occupano.
}\\
\hline
\end{tabular}
\end{center}
\label{Ricerca persone}
\end{table}

\section{Requisiti non funzionali}
\begin{table}[ht]
\caption{RNF: Tempi di risposta}
\begin{center}
\begin{tabular}{|p{3cm}|p{4cm}|p{5cm}|}
\hline
\textbf{RNF01}&\textbf{Applicazione Web}&\textbf{Tempi di Risposta}\\
\hline

\textbf{Descrizione}&\multicolumn{2}{|p{9cm}|}{
il software deve consetire all'utente di ottenere risposte in tempo reale. Nel caso in cui il seerver su cui \`e pubblicato possa essere accessibile in modalit\`a locale (LAN), il requisito fondamentale \`e di disporre di un computer con buone prestazioni e di qualit\`a di ricezione dati della rete buona. Per una buona operativit\`a da parte dell'utente \`e opportunuo che il 90\% delle transizioni siano processate in meno di 7 secondi e il 10\% in meno di 20 secondi.
}\\
\hline
\end{tabular}
\end{center}
\label{Ricerca persone}
\end{table}
\begin{table}[ht]
\caption{RNF: Visualizzazione di suggerimenti nella barra di ricerca}
\begin{center}
\begin{tabular}{|p{3cm}|p{4cm}|p{5cm}|}
\hline
\textbf{RNF02}&\textbf{Applicazione Web}&\textbf{Suggerimenti Persone}\\
\hline
\textbf{Input}&\multicolumn{2}{|p{9cm}|}{Scrittura di testo nella SearchBar.}\\
\hline
\textbf{Processo}&\multicolumn{2}{|p{9cm}|}{
L'utente visualizza suggerimenti di nomi di persone durante la scrittura di testo.
}\\
\hline
\textbf{Output}&\multicolumn{2}{|p{9cm}|}{
Elenco delle persone il cui nome o cognome comincia con il testo inserito nella SearchBar.
}\\
\hline
\end{tabular}
\end{center}
\label{Suggerimenti persone}
\end{table}

