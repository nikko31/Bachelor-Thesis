\subsection{Activity Diagram}
L’activity diagram UML è molto simile ai flowchart. Infatti, esso mostra i passi (chiamati, propriamente, attività), i punti decisionali e i rami che intervengono nel flusso di un programma. è utile per mostrare cosa accade in un processo di business o in un’operazione ed è considerato come parte integrante dell’analisi di sistema.

Un activity diagram viene disegnato per essere una vista semplificata di cosa accade durante un’operazione o un processo. Questo diagramma mostra gli stati di un oggetto e rappresenta le attività come frecce che si connettono agli stati. L’activity diagram serve per mettere in risalto le attività.

\begin{figure}[!htb]
\centering%
\includegraphics[scale=0.45]{ActDVisu.png}%
\caption{Activity diagram della funzione di visualizzazione di un'aula.}\label{fig:umlActDVisu}%
\end{figure}