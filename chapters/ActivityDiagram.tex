\subsection{Activity Diagram}
L’activity diagram UML è molto simile ai flowchart. Infatti, esso mostra i passi (chiamati, propriamente, attività), i punti decisionali e i rami che intervengono nel flusso di un programma. è utile per mostrare cosa accade in un processo di business o in un’operazione ed è considerato come parte integrante dell’analisi di sistema.

Un activity diagram viene disegnato per essere una vista semplificata di cosa accade durante un’operazione o un processo. Questo diagramma mostra gli stati di un oggetto e rappresenta le attività come frecce che si connettono agli stati. L’activity diagram serve per mettere in risalto le attività.

Nella figura\ref{fig:umlActDVisu} vengono messe in evidenza le attività eseguite per la visualizzazione delle informazioni di una stanza. Gli attori, ovvero i responsabili di una determinata attività, sono 4:
\begin{description}
\item[User:]è l'utente dell'applicazione; esso sceglie la mappa da visualizzare e, se tutto va a buon fine, può decidere se selezionare da essa una determinata aula o cambiare mappa.
\item[Client:]si occupa di mostrare all'utente le informazioni e le mappe tramite un'interfaccia chiara e pulita, e di gestire gli eventi inviando le informazioni al Server.
\item[Server:]trasferisce le mappe in formato SVG al Client e  interroga il database.
\item[Database:]si occupa del trasferimento delle informazioni richieste al server e del controllo di eventuali errori.
\end{description}

\begin{figure}[!htb]
\centering%
\includegraphics[scale=0.5]{ActivityDiag.png}%
\caption{Activity diagram della funzione di visualizzazione di un'aula.}\label{fig:umlActDVisu}%
\end{figure}
\FloatBarrier
Nel diagramma \ref{fig:umlActDSearch} si mostrano invece le attività svolte dagli attori sopra citati durante la fase di ricerca. 
\begin{figure}[!htb]
\centering%
\includegraphics[scale=0.5]{ActivityDiagSearchPersxml.png}
\caption{Activity diagram della funzione di ricerca.}\label{fig:umlActDSearch}%
\end{figure}
\FloatBarrier
