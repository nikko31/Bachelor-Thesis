\documentclass[a4paper,pt11,oneside]{book}
\usepackage[italian]{babel}
\usepackage[T1]{fontenc}
\usepackage[utf8]{inputenc}
\usepackage{listings}
\usepackage{color}
\usepackage[section]{placeins}
\usepackage{float} 
\usepackage{frontespizio}

\definecolor{dkgreen}{rgb}{0,0.6,0}
\definecolor{gray}{rgb}{0.5,0.5,0.5}
\definecolor{mauve}{rgb}{0.58,0,0.82}
\definecolor{light-gray}{gray}{0.25}

\lstset{
  language=Java,
  aboveskip=3mm,
  belowskip=3mm,
  showstringspaces=false,
  columns=flexible,
  basicstyle={\footnotesize\ttfamily},
  numberstyle={\tiny},
  numbers=left,
  numberblanklines=false,
  escapeinside=||,
  keywordstyle=\color{blue},
  commentstyle=\color{dkgreen},
  stringstyle=\color{mauve},
  breaklines=true,
  breakatwhitespace=true,
  tabsize=3
}
\linespread{1.5}
\usepackage{graphicx}
\graphicspath{ {images/} }
\title{
\huge
\textbf{Spazi ~Aule con Framework ~GWT}\\
{\includegraphics[scale=2]{university.jpg}}}
\author{Ferrari~Nico}
\begin{document}
\begin{frontespizio}

\Rientro{1.5cm}

\Margini{1.5cm}{1.5cm}{1.5cm}{1cm}

\Universita{Modena e Reggio Emilia}

%\Logo{c:/users/utente/Desktop/Tesidefimages/university.jpg}

\Facolta{Ingegneria Enzo Ferrari}

\Corso[Laurea]{Ingegneria Informatica}

\Annoaccademico{2015-2016}

\Titoletto{Tesi di laurea triennale}

\Titolo{Sviluppo di una webapp per la\\ gestione degli ambienti\\ universitari con framework GWT}

%\Sottotitolo{considerazioni....}

\Candidato[79083]{Nico Ferrari}

\Relatore{Prof. Nicola Bicocchi}

%\Correlatore{da inserire}

\end{frontespizio}

\hyphenpenalty=10
\exhyphenpenalty=10000
\tableofcontents
\listoffigures
\listoftables

%-------->Chapter

\chapter{Introduzione}

\begin{abstract}
Il lavoro di tesi svolto si pone come obiettivo lo sviluppo di una applicazione web che mette a disposizione degli utenti una visione chiara e completa dell'occupazione delle aule negli spazi universitari. L'applicazione si propone quindi di rendere un servizio sia basato sulla semplice e veloce consultazione delle informazioni riguardanti lo stato delle aule universitarie, sia la manipolazione di tali informazioni.
\end{abstract}

\newpage
%-------->Chapter
\chapter{Strumenti utilizzati}

\section{MySql}
MySQL è un Relational Database Management System (RDBMS), composto da un client con interfaccia a riga di comando e un server, entrambi disponibili sia per sistemi Unix e Unix-like che per Windows, anche se prevale un suo utilizzo in ambito Linux e Oracle Solaris.

Dal 1996 supporta la maggior parte della sintassi SQL e si prevede in futuro il pieno rispetto dello standard ANSI. 
MySQL e' il DBMS relazionale Open Source piu' diffuso al mondo (Questo è lo slogan ufficiale:  \textit{MySQL :The world’s most popular open source database}). 
L’aumento esponenziale della diffusione 
di MySQL è spiegabile con l’enorme 
successo del movimento open source.    
I principali punti di forza sono:  
\begin{itemize}
\item Gratis per l'utilizzo come open source.
Il prodotto è open source, quindi 
viene fornito oltre al prodotto anche il 
codice sorgente e questo consente all'utente di vedere come funzionano i programmi e di modificarli a seconda delle proprie esigenze. 
\item Eccezionale diffusione, soprattutto
 per le applicazioni web.  
\item Disponibile anche con una licenza commerciale e supporto tecnico.  
\item Leggero e di poco impatto sui server su cui viene installato . 
\item Semplice nell'utilizzo, nella configurazione e nell'amministrazione
\item Elevate prestazioni. 
\item Consente l'utilizzo di differenti storage engine tra cui scegliere.
\item Disponibile per una grande varietà di piattaforme. 
\\end{itemize}

\section{JAVA}
Il  linguaggio Java  \`e  un  linguaggio  di  programmazione  orientato  agli  oggetti,  creato  da  James 
Gosling  e  altri  ingegneri  di  Sun  Microsystems.  Il  gruppo  inizi\`o  a  lavorare  nel  1991,  il  linguaggio 
inizialmente si chiamava Oak. Il nome fu successivamente cambiato in Java a causa di un problema di  copyright  (il  linguaggio  di  programmazione  Oak  esisteva  gi\`a nel  1991).  Java  fu  annunciato ufficialmente il 23 maggio 1995 a SunWorld. La piattaforma di programmazione Java \`e fondata sul 
linguaggio  stesso,  sulla  Java  Virtual Machine  (JVM) e  sulle  API.  Java  \`e  un  marchio  registrato  di Sun Microsystems.
\subsection{Panoramica del linguaggio JAVA}
Java venne creato per soddisfare quattro scopi: 
\begin{itemize}
\item essere orientato agli oggetti
\item essere indipendente dalla piattaforma
\item contenere strumenti e librerie per il networking
\item essere progettato per eseguire codice da sorgenti remote in modo sicuro
\end{itemize}
\subsection*{Orientamento agli oggetti}
Java \`e un linguaggio object-oriented. La programmazione orientata agli oggetti (OOP, Object Oriented Programming) \`e un paradigma di programmazione che permette di definire oggetti software in grado di interagire gli uni con gli altri attraverso lo scambio di messaggi.  
La programmazione ad oggetti prevede di raggruppare in una zona circoscritta del codice sorgente (chiamata classe) la dichiarazione delle strutture dati e delle procedure che operano su di esse. Le classi, quindi, costituiscono dei modelli astratti, che a tempo di esecuzione vengono invocate per instanziare o creare oggetti software relativi alla classe invocata. Questi ultimi sono dotati di attributi (dati) e metodi (procedure) secondo quanto definito/dichiarato dalle rispettive classi.

La programmazione ad oggetti \`e particolarmente adatta nei contesti in cui si possono definire delle relazioni di interdipendenza tra i concetti da modellare (contenimento, uso, specializzazione). 

Tra i vantaggi della programmazione orientata agli oggetti abbiamo:
\begin{itemize}
\item fornisce un supporto naturale alla modellazione software degli oggetti del mondo reale o del modello astratto da riprodurre
\item permette una pi\`u facile gestione e manutenzione di progetti di grandi dimensioni
\item permette una pi\`u facile gestione e manutenzione di progetti di grandi dimensioni
\item l'organizzazione del codice sotto forma di classi favorisce la modularit\`a e il riuso di codice
\end{itemize}
\subsection*{Indipendenza dalla piattaforma}
L'indipendenza dalla piattaforma significa che l’esecuzione di programmi scritti  in  Java  deve  avere  un  comportamento  simile 
in contesti di esecuzione diversi. Per raggiungere questo obiettivo il codice Java viene compilato in un linguaggio intermedio bytecode. Il bytecode \`e un insieme di istruzioni altamente ottimizzate eseguibili dalla Java Virtual Machine(JVM), disegnata inizialmente come interprete per tale linguaggio.

La traduzione di un programma Java in bytecode rende molto piu semplice l'esecuzione di un programma in una vasta gamma di ambienti perch\`e solo la JVM deve essere implementata per ogni piattaforma.
\subsection*{Esecuzione sicura del codice remoto}
La piattaforma Java ha caratteristiche progettate per aumentare la sicurezza delle applicazioni Java:
\begin{itemize}
\item \textbf{La JVM:}\\
La JVM esegue una verifica del bytecode prima di eseguirlo per prevenire l'esecuzione di operazioni non sicure e per far rispettare vincoli a runtime. La piattaforma non permette ai programmi di eseguire alcune operazioni insicure e controlli manuali sull'allocazione e deallocazione della memoria.
\item \textbf{Il Security Manager:}\\
La piattaforma fornisce un security manager che permette agli utenti di eseguire codice bytecode inaffidabile in un ambiente \textit{sandbox} progettato per proteggerli da software pericoloso o scritto male evitando che tale codice abbia accesso ad alcune caratteristiche e API della piattaforma.
Il security manager permette di assegnare ai programmi Java una firma digitale in modo tale che l'utente possa scegliere di dare i privilegi a software con firma digitale valida proveniente da entità di fiducia.
\item \textbf{Le API:}\\
Viene fornita una serie di API orientate alla sicurezza, come algoritmi standard di crittografia,autenticazione e protocolli di comunicazione sicuri.
\end{itemize}

\subsection{La piattaforma JAVA}
La  piattaforma  Java  \`e  una  piattaforma  solo software eseguita sopra ad  una  piattaforma hardware di base che pu\`o essere un computer, una tv, un telefono cellulare, una smart card, ecc...
\\ La piattaforma Java \`e composta da due blocchi: 
\begin{itemize}
\item la Java Virtual Machine (JVM)
\item la Java Application Program Interface (API)
\end{itemize} 
La JVM \`e  la  base  della  piattaforma  Java, mentre la Java API \`e  una  collezione  di componenti software pronti all’uso per losvolgimento dei pi\`u disparati compiti.
\subsection*{Java Virtual Machine} 
La JVM consiste in:
\begin{description}
\item [Class loader:]
carica le classi che formano il bytecode, sia dell'applicazione Java, sia delle API Java necessarie per l'esecuzione da parte dell'interprete Java.

\item [Class verifier:]
controlla che il bytecode sia valido, che non superi i limiti superiori o inferiori dello stack, assicura non esegua aritmetica dei puntatori (che potrebbe potenzialmente portare ad una violazione di memoria). Se il bytecode passa tutti questi controlli, pu\`o essere eseguito dall'interprete.

\item [Interprete Java:] 
pu\`o essere di varie forme: pu\`o essere un modulo software che interpreta il bytecode in una sola volta oppure pu\`o fare uso di un compilatore just-in-time (JIT, o Just-In-Time compiler) che traduce a run-time il bytecode in codice nativo della macchina ospitante e lo salva in memoria durante l'esecuzione. \`E anche possibile utilizzare un sistema "misto", in cui il JIT viene applicato solo alle porzioni di codice del programma utilizzate pi\`u frequentemente, mentre il resto viene interpretato.
\end{description}

\subsection*{API Java}
Le API Java raccolgono una gran quantit\`a di componenti disponibili per scrivere applicazioni di qualsiasi genere. Per questo motivo la piattaforma Java \`e disponibile in tre configurazioni a seconda dell'uso che se ne vuole fare:
\begin{itemize}
\item \textbf{Standard Edition.} Fornisce API per le esigenze pi\`u comuni, che permette di scrivere applicazioni stand-alone, applicazioni client e server in un contesto di reti di computer, applicazioni per accesso a database, applicazioni per il calcolo scientifico e di altro tipo.
\item \textbf{Enterprise Edition.} Permette di scrivere applicazioni distribuite.
\item \textbf{Micro Edition.} Permette di scrivere applicazioni per i terminali mobili e, pi\`u in generale, per i dispositivi dotati di poche risorse computazionali (telefoni cellulari, palmari, smart cards ed altri).
\end{itemize}



\section{Framework Hibernate}
La gestione della persistenza \`e un argomento delicato nella progettazione di applicazioni software.
Lo sviluppo di applicazioni Object-Oriented che utilizzano un DBMS relazionale per immagazinare i dati in memoria secondaria \`e uno scenario molto comune. Ci\`o che rende complessa la costruzione dello strato di persistenza \`e il problema dell'\emph{object/relational impedence mismatch}, ovvero la discrepanza tra il paradigma Object-Oriented e il paradigma relazionale.\\
Fondamentalmente, gli oggetti fanno riferimento ad altri oggetti e perci\`o danno forma ad un grafo, a differenza degli gli schemi relazionali, che hanno invece una struttura tabellare e basati sull'algebra relazionale che definisce insiemi di tuple. La conversione di tuple in strutture a grafo spesso richiede molto tempo e pu\`o risultare difficile, infatti, viene anche descritta "\emph{Vietnam of Computer Science}".
Questo problema pu\`o esser diviso in pi\`u parti:
\begin{itemize}
\item \textbf{Struttura, Ereditariet\`a, Interfaccia e Polimorfismo}:\\
Una classe specifica gli attributi e i metodi che devono avere gli oggetti che appartengono ad essa e pu\`o anche far parte di una gerarchia (concetto di ereditariet\`a).
Il modello relazionale non contempla il concetto di “classe” di oggetti e non fornisce una analogia per le gerarchie, interfacce e polimorfismo.  
\item \textbf{Granularit\`a dei tipi di dato}:\\
Tipi di dati composti sono tipicamente  rappresentati nei linguaggi OO mediante classi di oggetti, mentre nel modello relazionale non si prevede alcun meccanismo per la definizione di tipi di dato composti. 
\item \textbf{Incapsulamento}:\\
Lo stato di un oggetto \`e incapsulato ed \`e accessibile attraverso i metodi. Lo stato di una riga di una tabella non contempla questo aspetto e pu\`o essere modificato in maniera diretta.
\item \textbf{Identit\`a}:\\
Gli oggetti esistono indipendentemente dal loro stato. Essi possono essere identici o uguali. Se due oggetti sono identici, essi sono lo stesso oggetto. Se sono uguali, essi contengono gli stessi valori. Nel modello relazionale, invece, le righe sono identificate solo dai valori che esse contengono.
\item \textbf{Associazioni}:\\
Il modello relazionale contempla solo un tipo di associazione: l'utilizzo di una chiave esterna, che si riferisce ad una chiave primaria di un'altra tabella. Il paradigma Object-Oriented contempla diverse tipologie di associazione: uno a uno, uno a molti, molti a molti.
\item \textbf{Coesione}:\\
Tutte le propriet\`a di un oggetto sono contenuto all'interno di esso. Invece, le relazioni che corrispondono ad una stessa entit\`a possono essere stati suddivisi in pi\`u tabelle (fase di ristrutturazione dello schema logico)
\end{itemize}
Il problema dell'impedence mismatch deve essere analizzato e risolto nel modo opportuno in fase di progettazione. 

Un altro aspetto molto importante nella gestione della persistenza \`e l'interazione tra lo strato di business logic e lo strato di persistenza.
In applicazioni Java la comunicazione con la base di dati avviene utilizzando API specifiche come ODB (Open Database Connectivity) o JDBC(Java Database Connectivity). 
Una delle strategie pi\`u valide e utilizzate \`e l'uso del pattern Data Access Object (DAO) per la creazione di oggetti che costituiscono uno strato di comunicazione con la base di dati. 
La realizzazione dei DAO avviene usando API JDBC
per rendere disponibili le operazioni CRUD(Create Read Update Delete) allo strato di logica applicativa. I DAO incapsulano l'accesso alla base di dati e permettono di gestire la maggior parte degli scenari ma la loro realizzazione necessita di una conoscenza dettagliata della base di dati.
Altri strategie fanno uso della serializzazione, di object oriented database systems oppure dell'Object-Relational Mapping che consiste nel mappare le classi di dominio dell'applicazione su una base di dati relazionale. 

\subsection{Object/Relational Mapping}
L'Object/Relational Mapping (ORM) \`e una tecnica di programmazione che favorisce l'integrazione di sistemi software aderenti al paradigma della programmazione orientata agli oggetti con sistemi RDBMS.
Attraverso questo mezzo ogni oggetto viene reso persistente nel database tramite l'inserimento di nuovi record i cui campi contengono i valori degli attributi dell'oggetto stesso. Si viene quindi a creare la relazione tra oggetto Java e tabella SQL. Una soluzione ORM consiste di quattro componenti caratteristici:

\begin{enumerate}
\item Una API per eseguire le operazioni CRUD sugli oggetti delle classi del modello.
\item  Un linguaggio  o una API  per specificare query che fanno riferimento alle classi e alle propriet\`a delle classi.
\item Uno srumento per la specifica del mapping mediante metadata.
\item Una tecnica per l'implementazione di ORM  per interagire con oggeti transazionali al fine di eseguire funzioni di ottimizzazione.
\end{enumerate}

L'associazione fisica tra la classe e la tabella viene ottenuta mediante l'utilizzo di file di descrizione (detti file di mapping), in cui si specificano le modalit\`a di conversione tra gli attributi dell'oggetto e i campi della tabella.\\In definitiva i principali vantaggi derivanti dall'uso di una soluzione ORM sono:

\begin{itemize}
\item Snellimento della parte di codice riservata alla persistenza dei dati.
\item Maggiore facilit\`a per quanto riguarda la manutenzione del codice grazie alla separazione tra il modello ad oggetti e il modello relazionale.
\item Performance pi\`u efficienti grazie alle numerose opzioni di ottimizzazione presenti.
\item Elevata portabilit\`a rispetto alla tecnologia DBMS utilizzata.
\end{itemize}

Uno degli esempi pi\`u noti di ORM per il linguaggio Java \`e Hibernate.
\subsection{Cos'\`e Hibernate}
Hibernate \`e un framework open source
con servizi ORM in Java. Hibernate ORM (fino alla versione 4.0 conosciuto come Hibernate Core), composto dalle API native di Hibernate e il suo motore, \`e disponibile alla versione 4.1, utilizzata in questa trattazione.
Il software \`e composto da API Java che permettono di gestire la persistenza con il mapping delle classi sulla base di dati e forniscono interfacce per l'accesso ai dati persistenti.
\subsection{Architettura Hibernate}
Facciamo riferimento alla figura sottostante per  trattare i componenti principali dell'architettura Hibernate.

\FloatBarrier
\begin{figure}[H]
\centering%
\includegraphics[scale=0.5]{hib-arch.png}%
\caption{Struttura di Hibernate}\label{fig:hibernate}%
\end{figure}

L'oggetto Configuration \`e di fondamentale importanza per il funzionamento del framework, in quanto contiene informazioni che riguardano:
\begin{itemize}
\item la connessione al database
\item il class mapping
\end{itemize}
la prima viene gestita mediante il file di configurazione \emph{hibernate.cfg.xml}, mentre il mapping tra le classi java e le tabelle del database, pu\`o avvenire mediante file di configurazione \textit{XML}, oppure, tramite l'inserimento diretto all'interno delle classi del codice inerente alla mappatura utilizzando lo stile Java Annotation.

\FloatBarrier
\begin{figure}[!htb]
\centering%
\includegraphics[scale=0.7]{hiber-arch2.jpg}%
\caption{Workflow Hibernate}\label{fig:hibernate2}%
\end{figure}

Questo oggetto viene creato una sola volta all'atto della inizializzazione della applicazione,

\begin{lstlisting}
Configuration config = new Configuration().configure("/it/resources/hibernate.cfg.xml");
\end{lstlisting}

e dopo che \`e stato utilizzato per la creazione di un altro oggetto, il SessionFactory, resta inutilizzato.

\begin{lstlisting}
SessionFactory sessionFactory = config.buildSessionFactory();
\end{lstlisting}

Quest'ultimo quindi, viene anch'esso creato una sola volta allo startup, ma a differenza di Configuration, viene utilizzato in tutta l'applicazione. L'oggetto SessionFactory \`e thread safe, ed \`e utilizzato da tutti i thread dell'applicazione, inoltre, siccome dipende da Configuration, che a sua volta si riferisce ad uno specifico database, potr\`a puntare solo a quel database. Nel caso di connessioni a database multipli, avremmo mulipli Configuration e quindi multipli SessionFactory, uno per ogni Configuration.
Il SessionFactory, come si evince dal nome, risulta essere uno stampo da cui creare gli oggetti Session, necessari ogni qualvolta occorre effettuare una interazione con il database. Quando con un oggetto di questo tipo viene aperta una sessione di lavoro,

\begin{lstlisting}
Session session = sessionFactory.openSession();
\end{lstlisting}

viene stabilita una connessione fisica con il database, e siccome non \`e thread safe, dopo aver effettuato le operazioni necessarie, occorre poi chiuderla manualmente, in modo da non mantenere connessioni aperte inutilmente.
Se durante una sessione di lavoro, occorre fare operazioni di modifica sul database, come inserimenti, aggiornamenti o cancellazioni, occorre gestirle come transazioni: a tal proposito l'oggetto Session viene a sua volta utilizzato per la creazione di oggetti Transaction. 
In Hibernate le transazioni sono gestite dal TransactionManager che permette allo sviluppatore di astrarsi dal livello sottostante (JDBC, JTA, ecc.) evitando di scrivere codice specifico. Infine Query e Criteria sono utilizzati per recuperare oggetti persistenti. Gli oggetti Query utilizzano SQL oppure HQL (Hibernate Query Language) per il recupero di dati dal database e per la creazione di oggetti, mentre Criteria utilizza oggetti per la costruzione e la esecuzione di una richiesta di recupero dati. 


\section{Framework GWT}
La necessit\`a di creare un'applicazione Web nasce dai molteplici vantaggi che le applicazioni RIA ( Rich Internet Application ) possiedono nei confronti delle tecnologie alternative. Infatti, rispetto alle applicazioni desktop, non richiedono installazione, gli aggiornamenti sono automatici, sono indipendenti dalla piattaforma utilizzata, pi\`u sicure in quanto girano nel ristretto ambiente del Web browser e maggiormente scalabili perch\`e la maggior parte del lavoro computazionale viene eseguito dal server.\\
Con l'avvento della tecnologia Ajax (Asynchronous JavaScript and XML), lo sviluppo di applicazioni Web si basa su uno scambio di dati in background fra Web browser e server, che
consente l'aggiornamento dinamico di una pagina Web senza esplicito ricaricamento da parte dell'utente. Purtroppo, scrivere applicazioni Ajax \`e molto complicato e perci\`o particolarmente esposto ad errori e bug; questo perch\`e JavaScript \`e un linguaggio piuttosto differente da Java e richiede molta pratica per lo sviluppo; il tutto \`e peggiorato dal fatto che JavaScript tende ad avere differenze in base al browser Web utilizzato, concentrando gli sforzi ed il tempo degli sviluppatori pi\`u sulla parte grafica che sulla logica applicativa. Google Web Toolkit (GWT) nasce proprio per risolvere questi problemi, fornendo un vero e proprio livello di astrazione che nasconde il codice JavaScript e provvede automaticamente ad uniformare le differenze tra i browser. \\
Rilasciato da Google nell'estate 2006 sotto licenza Apache, Google Web Toolkit \`e un set di tool open source che permette agli sviluppatori Web di creare e gestire complesse applicazioni fronted JavaScript scritte in Java. Il codice sorgente Java pu\`o essere compilato su qualsiasi piattaforma con i file Ant inclusi. I punti di forza di GWT sono la riusabilit\`a del codice, la possibilit\`a di realizzare pagine Web dinamiche mediante le chiamate asincrone di Ajax, la gestione delle modifiche, il bookmarking, l'internazionalizzazione e la portabilit\`a fra i differenti browser. 
Nello sviluppo dell'applicativo è stata usata l'ultima release stabile del framework, ovvero laversione 2.7.
\subsection{Architettura GWT 2.7}
GWT \`e caratterizzato da tre componenti base:
\begin{enumerate}
\item Un compilatore Java-JavaScript di alta qualit\`a.
\item La Java Runtime Enviroment (JRE) Emulation library.
\item Interfaccia utente
\end{enumerate}
\subsubsection*{Il compilatore}
Il compilatore Java-Javascript \`e il componente principale di GWT. Si occupa di prendere il codice Java 1.5 e produrre una versione equivalente in JavaScript ed incapsulare la varie differenze tra i browser.
Il compilatore esegue numerosi processi di ottimizzazione del codice, debugging, logging e generazione del codice.
La generazione del codice JavaScript pu\`o essere fatta con uno di questi tre stili:
\begin{description}
\item[-Offuscato]il codice \`e illeggibile, compatto e di piccole dimensioni, quindi \`e raccomandato solamente quando pubblichiamo in produzione la nostra applicazione.
\item[-Formattato]codice leggibile
\item[-Dettagliato]codice ricco di specificatori di package e signature complete: questo stile \`e molto comodo per ricavare informazioni del codice Java dalla console degli errori JavaScript.
\end{description}
\subsubsection*{JRE Emulation library}
Contiene le implementazioni in linguaggio JavaScript delle librerie Java standard maggiormente utilizzate(package java.lang.*,java.sql e java.util.*). Gli altri package(ad esempio java.io.*)sono molto limitati, includendo solo alcune interfaccie. Questa limitazione deriva dal fatto che il codice JavaScript generato da GWT viene eseguito su una sandbox del browser e stampanti.
\subsubsection*{libreria UI}
\`E una libreria User Interface contenente un insieme di interfacce e classi che permettono di disegnare le pagine web (ad es. bottoni, text boxes, immagini, ecc.). Questa \`e la libreria standard principale per creare applicazioni web-based basate su GWT.
\subsection{Comunicazione Client-Server in GWT}
\`E noto che quando si sviluppa una RIA, cosa fondamentale \`e la comunicazione tra il browser(client) e il server.
GWT fornisce differenti strade per comunicare con un server e il formato dei dati da utilizzare dipende dal server con cui si interagisce.
\subsubsection*{Remote Procedure Calls(GWT RPC)}
GWT RPC \`e un framework che permette di facilitare il passaggio di oggetti Java tra client e server (e anche viceversa) attraverso il protocollo HTTP.
E' possibile utilizzare il framework GWT RPC per rendere trasparenti le
chiamate alle servlet Java e lasciare a GWT il compito di prendersi cura dei
dettagli di basso livello come la serializzazione degli oggeti.
Il meccanismo GWT RPC pu\`o esser diviso in tre parti:
\begin{enumerate}
\item Il servizio eseguito sul server
\item Il codice client che invoca il servizio
\item Gli oggetti Java trasmessi tra client e server
\end{enumerate}

\begin{figure}[htbp]
\centering%
\includegraphics[scale=0.7]{RPC1.png}%
\caption{GWT RPC: interazione tra server e client mediante data object}\label{fig:rpc}}%
\end{figure}

Utilizzando GWT RPC tutte le chiamate effettuate dalla pagina HTML al server sono asincrone. Questo significa che le chiamate non bloccano il
client mentre attende una risposta dal server, ma viene eseguito il codice immediatamente successivo.
I vantaggi di effettuare chiamate asincrone rispetto alle pi\`u semplici (per gli
sviluppatori) chiamate sincrone, si riscontrano in una migliore esperienza per gli utenti finali.  innanzitutto, l'interfaccia utente \`e pi\`u reattiva; infatti,
a causa del fatto che nei browser Web il motore JavaScript \`e generalmente di
tipo single-thread, una chiamata sincrona al server genera un "blocco" fino alla conclusione della stessa, rovinando cos\`i l'esperienza dell'utente finale.
Altro vantaggio delle chiamate asincrone \`e che risulta possibile eseguire altri
lavori in attesa della risposta da parte del server; Ultimo vantaggio , ma non meno
importante, \`e che \`e possibile effettuare chiamate multiple al server nello
stesso tempo; tuttavia questo parallelismo risulta fortemente limitato dal
piccolo numero di connessioni che in genere i browser concedono alle singole
applicazioni.\\
I tipi di dato di scambio tra server e client devono essere innanzi tutto serializzabili e possono essere sostanzialmente dei seguenti tipi:
\begin{itemize}
\item Tipi primitivi Java.
\item I wrapper dei tipi primitivi
\item Un sottoinsieme degli oggetti Java Runtime Environemnt (JRE)
\item Qualsiasi tipo definito dall'utente a patto che sia serializzabile\footnote{Le chiamate alle GWT-RPC sono tra codice Javascript e Java e GWT prevede la serializzazione come parte del meccanismo RPC} 
(implementi l'interfaccia Serializable o IsSerializable di GWT)
\end{itemize}

\subsubsection*{Ricevendo dati JSON via HTTP}
Se l'applicazione comunica con un server che non pu\`o ospitare servlet Java, oppure con uno che utilizza gi\`a un'altro formato di dati come JSON o XML, si possono eseguire richieste HTTP per ottenere i dati. GWT fornisce classi HTTP generiche che possono essere utilizzate per fare le richieste, e classi XML e JSON client utilizzabili per processare le risposte.\\
Se si crea un'applicazione che richiede dati da uno o pi\`u web server remoti bisogna evitare le restrizioni SOP(Same Origin Policy)\footnote{Same Origin Policy \`e una misura di sicurezza del browser che limita il codice JavaScript client-side nell'interagire con risorse originate da nomi di dominio, porte e protocolli differenti.}.


\section{Tomcat}
Apache  Tomcat( o più semplicemente Tomcat) è  un’implementazione  delle  tecnologie  Java  Servlet  e  JavaServer  Pages. Le  specifiche  Java  Servlet  e  JavaServer  Pages  sono sviluppate  dal  “Java  Community Process”. Apache Tomcat è sviluppato in un ambiente aperto e partecipativo ed è rilasciato sotto  la  “Apache  Software  License”;  è destinato, inoltre,  ad  essere  una  collaborazione  dei più grandi sviluppatori di tutto il mondo. La  funzionalità  di  spicco  di  Tomcat  è  quella  del  “Web  Application  Server”,  ovvero  un server capace di gestire e supportare le pagine JSP e le servlet nel rispetto delle specifiche 3.1  (per  le  servlet)  e  2.3  (per  le  JSP). 

Tomcat è composto da tre parti:
\begin{description}
\item[Catalina:]è il contenitore di servlet Java di Tomcat. Catalina implementa le specifiche di Sun Microsystems per le servlets Java e le "JavaServer Pages (JSP, Pagine JavaServer).
\item[Coyote:]Coyote è il componente "connettore HTTP" di Tomcat. Supporta il protocollo HTTP 1.1 per il web server o per il contenitore di applicazioni. Coyote ascolta le connessioni in entrata su una specifica porta TCP sul server e inoltra la richiesta al Tomcat Engine per processare la richiesta e mandare indietro una risposta al client richiedente.
\item[Jasper]è il motore JSP di Tomcat. Tomcat 8.x utilizza Jasper 2, che è un'implementazione delle specifiche 2.3 delle Pagine JavaServer (JSP)[4]. Jasper analizza i file JSP per compilarli in codice Java come servlets (che verranno poi gestite da Catalina). Al momento di essere lanciato, Jasper cerca eventuali cambiamenti avvenuti ai file JSP e, se necessario, li ricompila.
\end{description}

\section{Grafica vettoriale}
La grafica vettoriale \`e una tecnica utilizzata in computer grafica per descrivere 
un'immagine. 
Un' immagine \`e descritta mediante un insieme di primitive geometriche che definiscono punti, linee, curve e poligoni ai quali possono essere attribuiti colori e anche sfumature. La grafica vettoriale presenta  vantaggi rispettio alla grafica raster\footnote{La grafica raster, o bitmap, \`e una tecnica per descrivere un'immagine digitale. \`E radicalmente diversa rispetto alla grafica vettoriale in quanto l’immagine \`e composta da una griglia di punti detti pixel, di forma quadrata.}; i principali vantaggi sono:
\begin{itemize}
\item possibilit\`a di esprimere i dati in una forma direttamente comprensibile ad un essere umano;
\item possibilit\`a di esprimere i dati in un formato che occupi (molto) meno spazio rispetto all'equivalente raster;
\item possibilit\`a di ingrandire l'immagine arbitrariamente, senza che si verifichi una perdita di risoluzione dell'immagine stessa.
\end{itemize}
\begin{figure}[!htb]
\centering%
\includegraphics[scale=0.4]{Vect.png}%
\caption{Differenza fra immagine di tipo raster e vettoriale.}\label{fig:imgVet}%
\end{figure}
\subsection{Scalable Vector Graphics}
Scalable Vector Graphics abbreviato in SVG, indica una tecnologia in grado di visualizzare oggetti di grafica vettoriale e, pertanto, di gestire immagini scalabili dimensionalmente.
Più specificamente si tratta di un linguaggio derivato dall'XML, cioè di un'applicazione del metalinguaggio posto a base degli sviluppi del Web da parte del consorzio W3C, che si pone l'obiettivo di descrivere figure bidimensionali statiche e animate.
Vediamo di elencare brevemente quelle che sono le caratteristiche principali di SVG:
\begin{description}
\item[È testuale:]questo implica che è possibile creare e modificare un file SVG con un semplicissimo editor di testo; inoltre si ha la possibilità di comprimere un file testuale in maniera molto efficiente favorendo così l’utilizzo di SVG in ambito Web;
\item[È vettoriale:]questo implica che, a differenza di quello che avviene nel caso di un formato grafico raster (Jpeg o Gif), è possibile scalare e zoommare a piacimento l’immagine SVG senza avere una perdita di qualità dell’immagine stessa;
\item[È open:]non è un formato proprietario, le specifiche ed i lavori del Working Group che si occupa di SVG sono liberamente consultabili sul sito del W3C;
\item[È un’applicazione XML:]questo permette di utilizzare, per lo sviluppo di applicazioni che manipolano file SVG, i numerosi strumenti di sviluppo già esistenti per XML;
\item[È interattivo:]attraverso un linguaggio di scripting è possibile rendere l’immagine SVG interattiva, ossia fare in modo che sia sensibile a certi eventi (come il click del mouse) e che compia determinate azioni in risposta a tali eventi;
\item[È dinamico:]infatti è possibile creare delle animazioni attraverso l’uso del linguaggio di animazione SMIL (Synchronized Multimedia Integration Language) anch’esso sviluppato dal W3C.
\end{description}

Gli oggetti grafici possono essere raggruppati in oggetti più comprensivi, muniti di attributi di stile e aggiunti ad oggetti grafici precedentemente costruiti e visualizzati. Un testo può far parte di un qualsiasi namespace XML sottoponibile ad una applicazione; questa possibilità consente di aumentare la ricercabilità e l'accessibilità delle immagini SVG.
Due programmi di grafica vettoriale open source e multipiattaforma che usano in maniera nativa il formato SVG sono Inkscape e Sodipodi.

\section{Maven}
Maven \`e un tool per l'automazione della compilazione usato principalmente per progetti Java.  Maven a ronta due aspetti della compilazione del software:
per prima cosa descrive come il software dev'essere compilato, e poi descrive le sue dipendenze.  A di erenza di tool pi\`u datati come Ant, utilizza convenzioni per la procedura di compilazione, in cui vanno speci cate solamente le eccezioni.  La configurazione del progetto va e ettuata in un  le XML, in cui viene descritto il progetto, le sue dipendenze, l'ordine di compilazione, le directory sorgenti, ed eventuali plugin.  Maven contiene target prede niti per task ben de niti, come ad esempio la compilazione e la pacchettizzazione.  Durante la risoluzione delle dipendenze, Maven scarica automaticamente le librerie e i plugin necessari da uno o pi\`u repository, e salva il tutto in una cache locale, che  pu\`o  essere  anche  aggiornata  con  artefatti  provenienti  da  progetti  locali.
Maven pu\`o anche essere usato per gestire progetti scritti in C\#, Ruby, Scala, e altri linguaggi.  Progetti alternativi come Gradle e
SBT (Scala Build Tool) non utilizzano XML ma mantengono i concetti chiave
introdotti da Maven.
I  progetti  Maven  vengono  configurati  tramite  un  Project  Object  Model (POM) file.  Generalmente, questa configurazione comprende il nome del progetto e le dipendenze su altri progetti.  Si possono anche configurare fasi individuali del processo di compilazione, implementate tramite plugin.  Ad esempio, si pu\`o configurare il plugin del compilatore in modo che venga utilizzata una versionefispeci ca di Java, oppure specificare che si vuole ottenere il pacchetto anche se alcune unit\`a di test falliscono.  I progetti pi\`u grossi dovrebbero essere suddivisi in diversi moduli (sottoprogetti), ognuno con il proprio POM, collegati ad un POM radice attraverso il quale si possono compilare tutti i moduli con un unico comando.  I  le POM possono anche ereditare la loro configurazione da altri  le POM. Di default, tutti i POM ereditano dal POM radice.  Il POM radice fornisce la configurazione di default, speci cando ad esempio le cartelle sorgente, i plugin, e cos\`i via.
La maggior parte delle funzionalit\`a di Maven \`e fornita dai plugin.  Esistono plugin per compilare,  testare,  e gestire i sorgenti,  avviare un web server, generare  le di progetto Eclipse, e molti altri ancora. I plugin vengono speci cati e configurati nell'apposita sezione del  le POM. Alcuni plugin di base sono inclusi di default in ogni progetto, con ragionevoli impostazioni di default.  Tuttavia la procedura risulterebbe scomoda se si dovesse ad ogni compilazione invocare ogni goal manualmente.  Per questo motivo Maven introduce  il  concetto  di  lifecycle,  ovvero  insiemi  di  goal  lanciati  in  modo
sequenziale.
La gestione delle dipendenze \`e una caratteristica chiave di Maven, e si fonda su un sistema che identi ca artefatti come librerie o moduli.  Le dipendenze dirette di un progetto vengono speci cate nel suo POM, ma vengono valutate e scaricate anche tutte le dipendenze transitive, che vengono poste nel repository locale.  Di default viene utilizzato il Maven 2 Central Repository per la ricerca delle librerie,  ma si possono configurare repository esterni all'interno del POM.
\begin{figure}[!htb]
\centering%
\includegraphics[scale=0.8]{maven.jpg}%
\caption{Maven}\label{fig:maven}%
\end{figure}

\section{Librerie}
\subsubsection*{Apache Batik}
\subsubsection*{Vectomatic lib-gwt-svg}
Lib-gwt-svg \`e una libreria opensource scaricabile dal sito www.vectomatic.org che permette di aggiungere le potenzialit\`a di SVG ad applicazioni GWT.

La stretta integrazione con GWT , tra cui: le risorse SVG (con convalida XML ), definizione di SVG in UiBinder , l'integrazione con gli eventi GWT, SVG Widget.

La stretta integrazione con Java , tra cui: le collezioni iterare , sottoclassi di elementi SVG , completa integrazione della documentazione W3C nei javadocs .

 licenza open source ( LGPL )
\newpage

%-------->Chapter
\chapter{Webapp GESTIONE AULE}
In questa sezione viene descritta l'architettura e le funzionalità del progetto, dividendo la parte server da quella client.
\begin{figure}[!htb]
\centering%
\includegraphics[scale=0.5]{MainPackDiag.png}%
\caption{Suddivisione principale del progetto}\label{fig:umlPackDiag}%
\end{figure}

Come si può notare dalla figura \ref{fig:umlPackDiag}, il progetto è stato suddiviso in due cartelle principali, una cartella contiene la parte di codice eseguita lato client dal browser web, mentre 
l’altra contiene la parte di codice eseguita lato server dall’application server Tomcat.

\section{Client Side}
La pagina iniziale è stata studiata per poter esser gestita in maniera semplice ed intuitiva dall'utente.  Tutti i contenuti sono stati suddivisi in aree  per  una  corretta  ed  intuitiva  visualizzazione. 
A  ciascuna sezione   sono   stati associati dei moduli dinamici (figura\ref{Screen:first}):
\FloatBarrier
\begin{itemize}
\item selezione mappa;
\item barra di ricerca persona;
\item visualizzazione mappa o risultati ricerca;
\item visualizzazione/modifica informazioni.
\end{itemize}

\begin{figure}[!htb]
\centering%
\includegraphics[scale=0.5]{FirstScreen.png}%
\caption{Organizzazione pagina iniziale}\label{Screen:first}%
\end{figure}
Un' importante fase nella progettazione di applicazioni software \`e quella della scelta di un'opportuna architettura, che definisce le linee guida allo sviluppo del progetto. Definire bene tale processo \`e utile sia per standardizzare il modello di sviluppo del progetto corrente, sia per le applicazioni
future.
L'impiego di un adeguato pattern porta numerosi vantaggi tra cui:
\begin{itemize}
\item Incrementa il riutilizzo del codice.
\item Facilita il lavoro in team e la pianificazione del progetto, dividendo
quest'ultimo in componenti indipendenti delegabili a gruppi di lavoro
differenti.
\item Aiuta la manutenzione del codice.
\item Aumenta la flessibilit\`a delle applicazioni e incrementa della scalabilit\`a.
\end{itemize}
Il Model-View-Presenter (MVP) \`e un pattern architetturale usato principalmente quando si vogliono creare User Interfaces. 
Il pattern MVP Separa la parte di gestione dei dati di un'applicazione dalla loro visualizzazione e manipolazione attraverso l'interfaccia utente. 
\begin{figure}[htbp]
\centering%
\includegraphics[scale=0.7]{MVP.png}%
\caption{Diagramma del Model-View-Presenter MVP}\label{fig:mvp}%
\end{figure}\\
Questo pattern \`e composto da tre elementi:
\begin{description}
\item[Model] \`e un'interfaccia che definisce i dati da visualizzare ed incapsula lo stato dell'applicazione.
\item[View] \`e un'interfaccia responsabile della visualizzazione dei
dati e delle informazioni, raccoglie gli input dell'utente e mai la manipolazione dei dati avviene in maniera diretta ma sempre attraverso un'interfaccia. Questo tipo di approccio consente di gestire facilmente eventuali modifiche alla GUI (Graphical User Interface) che non richiederanno mai l'aggiornamento del presenter.
\item[Presenter]  \`e colui che, oltre ad aggiornare la vista, interagisce con il model,che pu\`o essere identificato sia come lo stato di un oggetto che come dati persistenti in un'applicazione, in base alle richieste ricevute dalla view.
\end{description}
Nelle viste vi \`e una relazione uno a uno con i presenters e ci\`o permette a questi di osservare le loro viste e reagire agli eventi. Siccome le viste non possono interagire direttamente con altre viste, i presenters devono scambiare i dati con gli altri presenters. Per ottenere tale comunicazione  fra presenters, si fa uso dell' Event Bus, un oggetto in grado di trasmettere e filtrare le notifiche(figura \ref{fig:mvpEB}).

\FloatBarrier
\begin{figure}[!htb]
\centering%
\includegraphics[scale=0.5]{EBdiagram.png}%
\caption{Comunicazione tra unit\`a del software che non sono direttamente collegate tra loro. Ognuna di esse pu\`o inviare eventi all'Event Bus ed essere in "ascolto" di  particolari eventi.}\label{fig:mvpEB}%
\end{figure}

\FloatBarrier
\begin{figure}[!htb]
\centering
\includegraphics[scale=0.45,angle=270]{GWTappMVP.png}
\caption{Implementazione client seguendo pattern MVP}
\label{fig:mvpApp}
\end{figure}


\FloatBarrier
\subsection{Presentazione del sistema}
La pagina iniziale è stata studiata per poter esser gestita in maniera semplice ed intuitiva dall'utente.  Tutti i contenuti sono stati suddivisi in aree  per  una  corretta  ed  intuitiva  visualizzazione. 
A  ciascuna sezione   sono   stati associati dei moduli dinamici (figura\ref{Screen:first}):

\begin{itemize}
\item selezione mappa;
\item barra di ricerca persona;
\item visualizzazione mappa o risultati ricerca;
\item visualizzazione/modifica informazioni.
\end{itemize}

\begin{figure}[!htb]
\centering%
\includegraphics[scale=0.5]{FirstScreen.png}%
\caption{Organizzazione pagina iniziale}\label{Screen:first}%
\end{figure}

\FloatBarrier
\subsubsection*{Selezione mappa}
Nell'area "Selezione mappa" è possibile scegliere la mappa da visualizzare(tramite la selezione di un edificio e un suo piano) e la modalità in cui verrà presentata.

\subsubsection*{Barra di ricerca}
Questa barra permette di cercare le persone registrate nel database per ottenere le informazioni sulle stanze da loro occupate. Durante la tipizzazione del nome/cognome all'interno della barra di ricerca, vengono mostrati suggerimenti per velocizzare e rendere più accurata questo servizio (RF07).

\begin{figure}[!htb]
\centering%
\includegraphics[scale=0.5]{AdvicesScreen.png}%
\caption{Ricerca di persone.}\label{Screen:advices}%
\end{figure}
\FloatBarrier

\subsubsection*{visualizzazione mappa o risultati ricerca}
In quest'area sono possibili due casi:
\begin{enumerate}
\item viene mostrata la mappa in formao SVG nella modalità selezionata in "Selezione mappa";

\begin{figure}[!htb]
\centering%
\includegraphics[scale=0.3]{ShowMap.png}
\caption{Esempio di visualizzazione mappa}\label{Screen:showmap}%
\end{figure}
\FloatBarrier

\item viene fatta vedere una lista delle aule occupate dalle persone inserite nella "Barra di ricerca" con le relative informazione.

\begin{figure}[!htb]
\centering%
\includegraphics[scale=0.3]{SearchScreen.png}%
\caption{Esempio di risultato della ricerca}\label{Screen:search}%
\end{figure}
\FloatBarrier

\end{enumerate}

\subsubsection*{visualizzazione/modifica informazioni}
La visualizzazione e la modifica delle informazioni relative a stanze o persone selezionate avviene in questa sezione.

\begin{figure}[!htb]
\centering%
\includegraphics[scale=0.3]{RoomSelScreen.png}
\caption{Esempio di visualizzazione informazioni quando si seleziona una stanza.}\label{Screen:info1}%
\end{figure}

\begin{figure}[!htb]
\centering%
\includegraphics[scale=0.3]{ShowInfo2.png}%
\caption{Esempio di modifica informazioni di una stanza.}\label{Screen:info2}%
\end{figure}
\FloatBarrier


\FloatBarrier
\FloatingBarrier
\subsection*{Use Case Diagram}
Lo Use Case Diagram è una descrizione di un comportamento particolare di un sistema dal punto di vista dell'utente. Per gli sviluppatori, gli use case diagrams rappresentano uno strumento notevole: tramite tali diagramma, essi possono agevolmente ottenere una idea chiara dei requisiti del sistema dal punto di vista utente e quindi scrivere il codice senza timore di non aver recepito bene lo scopo finale.

Dal documento SRS si ricava il diagramma sottostante (figura\ref{fig:UseCDiagramFirst}) che mette in evidenza le funzioni principali accessibili dall'utente.
L'utente infatti, accedendo alla webapp, potrà visualizzare la mappa di un determinato piano in tre differenti modalità ( RF04, RF05, RF06).
Esso potrà anche cercare le aule occupate da una determinata persona inserendo il suo nome in una specifica barra di ricerca (RF07).
Dalla mappa si può selezionare un'aula e ottenere le informazioni relative ad essa (RF08) e successivamente modificare tali informazioni.

\begin{figure}[!htb]
\centering%
\includegraphics[scale=0.5]{UseCFirst.png}%
\caption{Use Case Diagram delle funzioni principali che l'utente può svolgere.}\label{fig:UseCDiagramFirst}%
\end{figure}



\subsection{Activity Diagram}
L’activity diagram UML è molto simile ai flowchart. Infatti, esso mostra i passi (chiamati, propriamente, attività), i punti decisionali e i rami che intervengono nel flusso di un programma. è utile per mostrare cosa accade in un processo di business o in un’operazione ed è considerato come parte integrante dell’analisi di sistema.

Un activity diagram viene disegnato per essere una vista semplificata di cosa accade durante un’operazione o un processo. Questo diagramma mostra gli stati di un oggetto e rappresenta le attività come frecce che si connettono agli stati. L’activity diagram serve per mettere in risalto le attività.

\begin{figure}[!htb]
\centering%
\includegraphics[scale=0.45]{ActDVisu.png}%
\caption{Activity diagram della funzione di visualizzazione di un'aula.}\label{fig:umlActDVisu}%
\end{figure}
\FloatBarrier
\subsection{Sequence Diagram}
il Sequence Diagram è un diagramma che illusta le interazioni tra gli oggetti disponendole lungo una sequenza temporale. In particolare mostra gli oggetti che partecipano alla interazione e la sequenza dei messaggi cambiati.
Un diagramma di sequenza rappresenta le successioni temporali ma non le relazioni tra gli oggetti, descrivendo così il comportamento dinamico del sistema.
\begin{figure}[!htb]
\centering%
\includegraphics[scale=0.5]{ActivityDiagShowRoom.png}%
\caption{Sequence diagram della funzione di visualizzazione delle informazioni di un'aula.}\label{fig:umlSeqDRoomInfo}%
\end{figure}
\FloatBarrier
In questo diagramma (figura\ref{fig:umlSeqDRoomInfo}) viene mostrata la procedura per la visualizzazione delle informazioni relative ad una determinata aula. \\
Vengono messe in risalto le funzioni specifiche delle View e dei relativi Presenter all'interno del client, e come questi ultimi interagiscono con il Server.\\
Per prima cosa l'utente deve selezionare l'edificio ed il relativo piano che si intende visualizzare e successivamente la modalità di visualizzazione "Visualizzazione" premendo in fine il bottone "Cerca". La "View" invierà quindi al relativo "Presenter" i dati selezionati che, tramite la funzione RequestBuild(), richiederà al server la mappa desiderata. Se la trasmissione va a buon fine verrà aggiornata la mappa nella "View", permettendo all'utente di selezionare un'aula  della mappa. Dopo aver acquisito le informazioni dell'aula selezionata dalla "View", il "Presenter", richiederà al server in maniera asincrona tramite RPC le informazioni di tale aula.

\begin{figure}[!htb]
\centering%
\includegraphics[scale=0.5]{SeqDiagSearchPers.png}%
\caption{Sequence diagram della funzione di ricerca delle occupazioni aule di determinate persone.}\label{fig:umlSeqDSearchPerson}%
\end{figure}
\FloatBarrier
Nel diagramma \ref{fig:umlSeqDSearchPerson} vengono invece mostrate le iterazioni durante la funzione di ricerca. Notiamo che l'utente può scegliere più nomi di persone digitandoli nella barra di ricerca e sfruttando i vari suggerimenti. Le richieste verrano effettuate al server tramite chiamate RPC e la View si occupera di creare una lista con i risultati ricevuti.


\section{Server Side}
\subsection{Comunicazione con server}
\FloatBarrier
Utilizzando GWT-RPC, i nostri oggetti del modello vengono automaticamente serializzati quando vengono utilizzati come parametri per una chiamata RPC. Le uniche restrizioni sono date dall’obbligatorietà di implementare 
l’interfaccia \emph{Serializable} e dalla presenza di un costruttore senza argomenti. 

\begin{figure}[!htb]
\centering%
\includegraphics[scale=0.5]{AnatomySVG.png}%
\caption{GWT-RPC }\label{fig:GWT-RPC}%
\end{figure}
\FloatBarrier
Il primo passo per l’attuazione del servizio RPC è quello di dichiarare l’interfaccia di servizio che estenda l’interfaccia \emph{RemoteService} ed elenchi tutti i metodi RPC.
Successivamente definire una classe per implementare il codice server-side che estenda la classe \emph{RemoteServiceServlet} ed implementi l’interfaccia creata precedentemente. 
Non rimane che implementare la versione asincrona dell'interfaccia per l’applicazione client poiché ogni metodo chiamato deve essere asincrono.

\begin{figure}[!htb]
\centering%
\includegraphics[scale=0.55]{rpcPack.png}%
\caption{Classi utilizzate per il servizio RPC. }\label{fig:RPCPack}%
\end{figure}
\FloatBarrier
\subsection{Database e mapping}

L'applicazione usa come supporto un database relazionale contenente i dati rigurdanti le aule dei vari edifici universitari e del personale.
\`E fondamentale però mettere in evidenza con un adeguato schema Entità/Relazione le tabelle utilizzate all’interno del progetto e create al fine del corretto funzionamento del 
sistema.

\begin{figure}[!htb]
\centering
\includegraphics[scale=0.60]{databaseER.png}
\caption{Diagramma E/R del database}\label{fig:database}
\end{figure}

Facendo riferimento alla figura \ref{fig:database}, descriviamo le entità fondamentali definite nel modello in esame:
\FloatBarrier
\begin{description}
\item[Room]\`e l'entit\`a principale. Essa contiene le informazioni riguardanti una specifica stanza di un determinato piano di un edificio. 
\item[Building]rappresenta l'edificio, caratterizzato da un numero univoco e dotato di un nome.
\item[Person]rappresenta il personale. Fondamentale per l'applicazione sono il ruolo di una persona (ad esempio docente e ricercatore) e le date di fine e inizio contratto lavorativo.
\item[OccupyRoom]associa una persona a una o pi\`u stanze diverse e  servir\`a ad ottenere una lista delle persone che occupano una stanza.
\end{description}

\FloatBarrier
\begin{figure}[!htb]
\centering
\includegraphics[scale=0.55]{diagram.png}
\caption{Modello Relazionale del database}\label{fig:databaseRelaz}
\end{figure}
\FloatBarrier
Dopo aver sviluppato il modello relazionale(figura \ref{fig:databaseRelaz}), tramite una metodologia bottom up, sono stati interfacciati oggetti Java con il database relazionale attraverso file di mappatura XML, uno per ogni classe che si vuole rendere persistente.\\ 

\begin{figure}[!htb]
\centering
\includegraphics[scale=0.55]{packageMapping.png}
\caption{Package contenente i file di mappatura e le corrispondenti classi Java.}\label{fig:mappingPack}
\end{figure}
\FloatBarrier
Questi file di mappatura sono caratterizzati  dalle dualità classe-tabella e attributo-colonna.
Per inviare le informazioni al client non si possono sfruttare queste classi utilizzate per Hibernate. Infatti, nonostante implementino Serializable, vengono interpretate in modo particolare dal compilatore (la libreria Javassist si occupa di riscrivere il bytecode di tali oggetti Hibernate) e quindi, quando devono essere serializzate o deserializzate per essere trasmesse al client, il meccanismo GWT RPC non sa come trattarle e le rifiuta. Per risolvere questo problema si fa uso di classi DTO (Data Transfer Object). Le classi DTO sono delle semplici classi contenenti esclusivamente i dati che si vogliono memorizzare senza la logica per la persistenza aggiunta da Hibernate; queste sono quindisfruttabili a lato client e utilizzate poi per "costruire" le classi Hibernate.
\begin{figure}[!htb]
\centering
\includegraphics[scale=0.45]{dtoScreen.png}
\caption{Classi DTO}\label{fig:dtoPack}
\end{figure}
\newpage

\chapter{Conclusioni}
Mi ritengo soddisfatto del progetto di tesi svolto.
Innanzitutto, sono stati affrontati
problemi pratici che riguardano la produzione di software, sia dal lato progettuale sia da quello implementativo.
L'adozione di frameworks è risultata molto efficacie, rendendo lo sviluppo del software più semplice permettendo la creazione di una struttura  adattabile e flessibile risolvendo in tal modo numerose problematiche.\\
GWT ha permesso di trasformare trasparentemente e automaticamente il codice prodotto in una RIA Ajax con tanto di client che lavora in modalità asincrona direttamente dentro il browser senza preoccuparsi del codice JavaScript.
Inoltre, grazie al pattern design adottato, abbiamo potuto lavorare efficacemente in team. \\
Hibernate ha permesso di ridurre significativamente i tempi  di sviluppo altrimenti impiegati in attività manuali di gestione dei dati in SQL e JDBC risolvendo problemi legati all'impedence mismatch e riducendo significativamente la quantità di codice per la persistenza rendolo altamente manutenibile.
La scelta di utilizzare immagini vettoriali SVG ha permesso un ampia manipolazione di esse grazie all'ausilio di librerie open source e rendendo l'interfaccia con l'utente intuitiva ed esteticamente gradevole.

\newpage

\chapter{Appendice A}
\section{Software Requirements Specification}
\subsection*{Requisiti funzionali}
\begin{table}[ht]
\caption{RF: Selezione Edificio}
\begin{center}
\begin{tabular}{|p{3cm}|p{4cm}|p{5cm}|}
\hline
\textbf{RF01}&\textbf{Applicazione Web}&\textbf{Selezione edificio}\\
\hline
\textbf{Input}&\multicolumn{2}{|p{9cm}|}{
Selezione dell'edificio desiderato. }\\
\hline
\textbf{Processo}&\multicolumn{2}{|p{9cm}|}{
L'utente pu\`o selezionare l'edificio desiderato da una ListBox.
}\\
\hline
\textbf{Output}&\multicolumn{2}{|p{9cm}|}{
Nessun output particolare.
}\\
\hline
\end{tabular}
\end{center}
\label{Selezione Edificio}
\end{table}
\begin{table}[ht]
\caption{RF: Selezione piano dell'edificio}
\begin{center}
\begin{tabular}{|p{3cm}|p{4cm}|p{5cm}|}
\hline
\textbf{RF02}&\textbf{Applicazione Web}&\textbf{Selezione piano}\\
\hline
\textbf{Input}&\multicolumn{2}{|p{9cm}|}{
Selezione del piano desiderato. }\\
\hline
\textbf{Processo}&\multicolumn{2}{|p{9cm}|}{
L'utente pu\`o selezionare il piano desiderato di un determinato edificio da una ListBox.
}\\
\hline
\textbf{Output}&\multicolumn{2}{|p{9cm}|}{
Nessun output particolare.
}\\
\hline
\end{tabular}
\end{center}
\label{Selezione Piano}
\end{table}
\begin{table}[ht]
\caption{RF: Selezione Modalit\`a di visualizzazione}
\begin{center}
\begin{tabular}{|p{3cm}|p{4cm}|p{5cm}|}
\hline
\textbf{RF03}&\textbf{Applicazione Web}&\textbf{Selezione Modalit\`a}\\
\hline
\textbf{Input}&\multicolumn{2}{|p{9cm}|}{
Selezione della modalit\`a di visualizzazione desiderata. }\\
\hline
\textbf{Processo}&\multicolumn{2}{|p{9cm}|}{
L'utente pu\`o selezionare una tra le modalit\`a di visualizzazione disponibili da una particolare ListBox.
}\\
\hline
\textbf{Output}&\multicolumn{2}{|p{9cm}|}{
Nessun output particolare.
}\\
\hline
\end{tabular}
\end{center}
\label{Visualizzazione Modalit\`a}
\end{table}
\begin{table}[ht]
\caption{RF: Visualizzazione della mappa del piano selezionato}
\begin{center}
\begin{tabular}{|p{3cm}|p{4cm}|p{5cm}|}
\hline
\textbf{RF04}&\textbf{Applicazione Web}&\textbf{Visualizzazione Mappa}\\
\hline
\textbf{Input}&\multicolumn{2}{|p{9cm}|}{
Premere bottone "Enter" }\\
\hline
\textbf{Processo}&\multicolumn{2}{|p{9cm}|}{
L'utente, dopo aver selezionato l'edificio e il piano desiderati e aver selezionato la modalit\`a "Visualizzazione", cliccher\`a su un determinato bottone per visualizzare la corrispondente mappa.
}\\
\hline
\textbf{Output}&\multicolumn{2}{|p{9cm}|}{
Mappa in modalit\`a visualizzazione.
}\\
\hline
\end{tabular}
\end{center}
\label{Visualizzazione mappa}
\end{table}
\begin{table}[ht]
\caption{RF: Distribuzione Spazi}
\begin{center}
\begin{tabular}{|p{3cm}|p{4cm}|p{5cm}|}
\hline
\textbf{RF05}&\textbf{Applicazione Web}&\textbf{Distribuzione Spazi}\\
\hline
\textbf{Input}&\multicolumn{2}{|p{9cm}|}{
Premere bottone "Enter".}\\
\hline
\textbf{Processo}&\multicolumn{2}{|p{9cm}|}{
L'utente, dopo aver selezionato l'edificio e il piano desiderati e aver selezionato la modalit\`a "Distribuzione Spazi", cliccher\`a su un determinato bottone per visualizzare la corrispondente mappa .
}\\
\hline
\textbf{Output}&\multicolumn{2}{|p{9cm}|}{
mappa che illustra la corrispondente distribuzione degli spazi sfruttando una specifica colorazione delle singole stanze.
}\\
\hline
\end{tabular}
\end{center}
\label{Distribuzione Spazi}
\end{table}
\begin{table}[ht]
\caption{RF: Visualizzazione termine attivit\`a lavorativa}
\begin{center}
\begin{tabular}{|p{3cm}|p{4cm}|p{5cm}|}
\hline
\textbf{RF06}&\textbf{Applicazione Web}&\textbf{Termine Attivit\`a Lavorativa}\\
\hline
\textbf{Input}&\multicolumn{2}{|p{9cm}|}{
Premere bottone "Enter" }\\
\hline
\textbf{Processo}&\multicolumn{2}{|p{9cm}|}{
L'utente, dopo aver selezionato l'edificio e il piano desiderati e aver selezionato la modalit\`a "Termine attivit\`a lavorativa", cliccher\`a su un determinato bottone per visualizzare la corrispondente mappa.
}\\
\hline
\textbf{Output}&\multicolumn{2}{|p{9cm}|}{
Mappa in modalit\`a Termine attivit\`a lavorativa dove \`e possibile consultare le scadenze dei contratti lavorativi delle persone che occupano le aule.
}\\
\hline
\end{tabular}
\end{center}
\label{Visualizzazione termine attivit\`a lavorativa}
\end{table}
\begin{table}[ht]
\caption{RF: Ricerca persone}
\begin{center}
\begin{tabular}{|p{3cm}|p{4cm}|p{5cm}|}
\hline
\textbf{RF07}&\textbf{Applicazione Web}&\textbf{Ricerca Persone}\\
\hline
\textbf{Input}&\multicolumn{2}{|p{9cm}|}{ Scrivere il nome delle persone da cercare e premere bottone "Search". }\\
\hline
\textbf{Processo}&\multicolumn{2}{|p{9cm}|}{
L'utente, dopo aver scritto il nome delle persone che vuole cercare nell'apposita SearchBar, cliccher\`a su un determinato bottone per visualizzare le informazioni riguardanti l'occupazione aule di tali persone.
}\\
\hline
\textbf{Output}&\multicolumn{2}{|p{9cm}|}{
Elenco delle aule occupate dalle persone ricercate
}\\
\hline
\end{tabular}
\end{center}
\label{Ricerca persone}
\end{table}
\begin{table}[ht]
\caption{RF: Visualizzazione informazioni stanza}
\begin{center}
\begin{tabular}{|p{3cm}|p{4cm}|p{5cm}|}
\hline
\textbf{RF08}&\textbf{Applicazione Web}&\textbf{Visualizzazione Informazioni Stanza}\\
\hline
\textbf{Input}&\multicolumn{2}{|p{9cm}|}{Premere su una stanza sulla mappa.}\\
\hline
\textbf{Processo}&\multicolumn{2}{|p{9cm}|}{
L'utente clicca su una stanza della mappa e verranno visualizzate le ralative informazioni.
}\\
\hline
\textbf{Output}&\multicolumn{2}{|p{9cm}|}{
Elenco delle informazioni relative alla stanza e l'elenco delle persone che la occupano.
}\\
\hline
\end{tabular}
\end{center}
\label{Ricerca persone}
\end{table}

\section{Requisiti non funzionali}
\begin{table}[ht]
\caption{RNF: Tempi di risposta}
\begin{center}
\begin{tabular}{|p{3cm}|p{4cm}|p{5cm}|}
\hline
\textbf{RNF01}&\textbf{Applicazione Web}&\textbf{Tempi di Risposta}\\
\hline

\textbf{Descrizione}&\multicolumn{2}{|p{9cm}|}{
il software deve consetire all'utente di ottenere risposte in tempo reale. Nel caso in cui il seerver su cui \`e pubblicato possa essere accessibile in modalit\`a locale (LAN), il requisito fondamentale \`e di disporre di un computer con buone prestazioni e di qualit\`a di ricezione dati della rete buona. Per una buona operativit\`a da parte dell'utente \`e opportunuo che il 90\% delle transizioni siano processate in meno di 7 secondi e il 10\% in meno di 20 secondi.
}\\
\hline
\end{tabular}
\end{center}
\label{Ricerca persone}
\end{table}
\begin{table}[ht]
\caption{RNF: Visualizzazione di suggerimenti nella barra di ricerca}
\begin{center}
\begin{tabular}{|p{3cm}|p{4cm}|p{5cm}|}
\hline
\textbf{RNF02}&\textbf{Applicazione Web}&\textbf{Suggerimenti Persone}\\
\hline
\textbf{Input}&\multicolumn{2}{|p{9cm}|}{Scrittura di testo nella SearchBar.}\\
\hline
\textbf{Processo}&\multicolumn{2}{|p{9cm}|}{
L'utente visualizza suggerimenti di nomi di persone durante la scrittura di testo.
}\\
\hline
\textbf{Output}&\multicolumn{2}{|p{9cm}|}{
Elenco delle persone il cui nome o cognome comincia con il testo inserito nella SearchBar.
}\\
\hline
\end{tabular}
\end{center}
\label{Suggerimenti persone}
\end{table}


\newpage

\chapter{Appendice B}
%-------database
\section*{Connessione e script database}
Durante la fase di sviluppo dell'applicazione, per connettersi al database remoto è stato necessario, per motivi di sicurezza, eseguire un tunneling SSH tramite il comando da terminale: 
\begin{lstlisting}
ssh -L localport:host:hostport user@ssh_server -N 
\end{lstlisting}
Lo script \emph{database.sql} usato per creare il database si trova nella directory \emph{AuleProjSVGWT/src} come mostrato in figura \ref{fig:dbSQL}.

\begin{figure}[!htb]
\centering%
\includegraphics[scale=0.5]{screenSQL.png}%
\caption{Cartella contenente script del database.}\label{fig:dbSQL}%
\end{figure}
\FloatBarrier
\section*{Creazione mappe e colorazione}
Le piantine dei locali universitari non possono essere inserite direttamente nel server perch\`e sono in un formato che non \`e compatibile col programma.Lo scopo di questa appendice \`e quello di spiegare la procedura da seguire per  ottenere dei file adeguati alle funzionalit\`a dell'applicazione.\\\\Le immagini fornite dal dipartimento sono in formato \textit{dwg}, quindi la prima operazione da effettuare \`e quella di convertirle in \textit{svg} e salvarle con questo nome: "\textit{nome\_edificio-numero\_piano.svg}".In rete esistono diversi software che consentono di cambiare l'estensione del file, nel nostro caso \`e stato utilizzato AIGraph CAD Viewer(versione 3.3.1 o superiore).Un altro metodo consiste nel trasformare i file \textit{dwg} in \textit{dxf}, importarli su Inkscape, per poi salvarli come \textit{svg}.Tra le due alternative \`e stata adottata la prima perch\`e ha permesso di generare figure con una migliore qualit\`a grafica.\\\\Nel file ottenuto le stanze non sono ancora elementi grafici indipendenti, perci\`o se si vogliono gestire eventi su ognuna di esse \`e necessario inserire, all'interno del file, delle forme geometriche che le racchiudano.Questa operazione, molto onerosa in termini di tempo, pu\`o essere velocizzata se si utilizza Inkscape.Le forme geometriche inserite con questo tool devono essere di tipo \textit{"rect"} o \textit{"path"}, il motivo di questa operazione sar\`a trattato nella parte in cui si parla del codice Java.Dopo aver creato il profilo di una stanza bisogna selezionare l'opzione \textit{object properties}(come avviene in figura \textbf{\ref{fig:TUT_ink}}) e introdurre nei tre riquadri disponibili la stringa 
\\
"\textit{nome\_edificio-numero\_piano-numero\_stanza}": l'inserimento di questi parametri, in particolar modo dell'ID, servir\`a al software per risalire alla stanza.Questo perch\`e i singoli termini della stringa hanno una corrispondenza con i dati inseriti nel database:  \\
\textit{nome\_edificio} \`e associato a \textit{building\_name} della tabella "building", invece \textit{numero\_piano} e \textit{numero\_stanza} equivalgono a \textit{room\_floor} e \textit{room\_number} della tabella "room".Dato che la loro combinazione rappresenta un'\textit{UNIQUE KEY} \`e possibile sfruttarla per risalire ad un unica istanza di "room".
\begin{figure}[H]
\centering
\fbox{\includegraphics[width=150mm,height=85mm]{esempioInkscape.png}}
\caption{Inkscape: inserimento stanza cliccabile}
\label{fig:TUT_ink}%
\end{figure}
\noindent\\Infine \`e necessario un ultimo passaggio: con un editor di testo o con un IDE qualsiasi(con \textit{encodig:"UTF-8"}) devono essere modificate e inserite alcune stringhe nel file.Per prima cosa bisogna controllare che siano presenti i seguenti attributi all'interno del tag \textit{<svg>}.
\begin{lstlisting}
<?xml version="1.0" encoding="UTF-8" standalone="no"?>
<svg
	..
	xmlns:dc="http://purl.org/dc/elements/1.1/"
	..
	xmlns:svg="http://www.w3.org/2000/svg"
	xmlns="http://www.w3.org/2000/svg"
	..
	version="1.1"
	..
	sodipodi:docname="materiali-1.svg"
\end{lstlisting}
\`E fondamentale che sia inserito anche all'interno il nome del file, e che la versione dello \textit{standard} sia la "\textbf{1.1}".\\Se non vengono rispettati questi dettagli possono insorgere degli errori nelle librerie che lavorano con le immagini.\\Di seguito \`e riportato un esempio di come deve apparire il codice di una stanza.
\subsubsection*{ANDROID}
\begin{lstlisting}
<path
   style="fill:none;stroke:none;fill-opacity:0.5;stroke-width:0.99871397px;stroke-linecap:butt;stroke-linejoin:miter"
   d="m 247.97113,203.91145 -10.22383,-88.12825 -37.94946,5.39561 10.5704,87.76853 z"
   id="materiali-1-4"
   inkscape:connector-curvature="0"
   inkscape:label="materiali-1-4"><title
     id="title30939">materiali-1-4</title>
</path>
\end{lstlisting}
Le stringhe \textbf{"fill:none"},\textbf{"stroke:none"} e \textbf{"fill-opacity:0.5"} devono sempre essere presenti tra le prorpiet\`a di \textit{style}, senza queste non sarebbe possibile effettuare la colorazione.Gli altri attributi invece possono essere trascurati, l'unica eccezione si ha per quelli che contengono la parola \textit{"gradient"}, in questo caso \`e consigliabile eliminarli perch\`e potrebbero creare delle sfumature nel colore.\\Come \`e stato detto in precedenza il codice relativo ad un stanza deve essere contenuto in elementi di tipo \textit{<path>} o \textit{<rect>}, se ne vengono utilizzati altri \`e necessario riadattare anche la parte Java.
\subsubsection*{GWT}
Va precisato che le librerie usate per GWT non supportano l'attributo \textbf{"none"}, dunque, essendo il numero di immagini limitato e la loro dimensione contenuta, si \`e deciso di utilizzare due file indipendenti per ogni ogni piantina: uno per Android nella cartella \textit{res/ImageAndroid} e uno per GWT nella cartella \textit{res/ImageGWT}.Di seguito \`e mostrato il codice riadattato per GWT.
\begin{lstlisting}
<path
   style="fill:transparent;stroke:transparent;fill-opacity:0.5;stroke-width:0.99871397px;stroke-linecap:butt;stroke-linejoin:miter"
   d="m 247.97113,203.91145 -10.22383,-88.12825 -37.94946,5.39561 10.5704,87.76853 z"
   id="materiali-1-4"
   inkscape:connector-curvature="0"
   inkscape:label="materiali-1-4"><title
   id="title30939">materiali-1-4</title>
</path> 
\end{lstlisting}
Come si pu\`o notare l'unica differenza sta nella sostituzione di \textit{none} con \textit{transparent}.
\subsubsection*{CODICE JAVA}
In questa sezione vengono analizzate le parti di codice Java che servono per effettuare lavorare sulle immagini.Nel primo riquadro \`e esposto il metodo con cui ottenere la lista delle stanze all'interno di una piantina.\\Grazie alla stringa \textit{"NOME\_FILE"} \`e possibile sapere su quale file effettuare la ricerca.\\Se si osserva attentamente, nelle righe 9-10, si pu\`o notare che l'indagine avviene solo per elementi di tipo "path" e "rect", nel caso venissero usate altre forme geometriche \`e necessario aggiungerle in questa zona del codice. 
\begin{lstlisting}
String fullPath = servletContext.getRealPath(path);
URI uri = new File(fullPath+"/" + NOME_FILE + ".svg").toURI();
SVGMetaPost converter;
ArrayList<String> room = new ArrayList<>();
	try{
    	converter = new SVGMetaPost(uri.toString());
    	Document doc = converter.getSVGDocument();
    
    	NodeList n = doc.getElementsByTagName("rect");
    	NodeList p = doc.getElementsByTagName("path");
    	for(int i =0 ;i<n.getLength();i++){
    		if(((Element) n.item(i)).getAttribute("id").contains(NOME_FILE)){
            	        room.add(((Element) n.item(i)).getAttribute("id"));
			}
		}
    	for(int i =0 ;i<p.getLength();i++) {
    		if(((Element) p.item(i)).getAttribute("id").contains(NOME_FILE)){
    			room.add(((Element) p.item(i)).getAttribute("id"));
    		}
    	}
	}catch (IOException e){
		e.printStackTrace();
	}		
	
return room;

}
\end{lstlisting}
\noindent\\Nella seconda area viene presentato lo pseudocodice per ottenere la variazione dei colori nelle piantine. 
\begin{lstlisting}
per ogni stanza{
	sum = somma dei pesi dei ruoli di ogni persona che occupa la stanza;
	dim = la dimensione in metri quadri della stanza;

	if ( dim == 0 e sum != 0 ){
		//colora la stanza di ROSSO;
	}
	else if( dim == sum e dim != 0 ){
		//colora la stanza di VERDE;
	}
	else if( sum > dim e dim != 0 ){
		//colora la stanza con una gradazione di ROSSO
	}
	} else if (sum < dim e dim != 0 ){
		//colora la stanza con una gradazione di BLU
	}

}
\end{lstlisting}
Per avere le gamme di ROSSO e BLU si opera in questo modo:
\\\textbf{Spazio disponibile superato}\\Si effettua questa divisione tra \textit{dim} e \textit{sum}.Il numero che otteniamo, compreso tra 0 e 1, viene moltiplicato per 200 e trasformato in numero esadecimale.Il valore esadecimale viene poi usato per creare una stringa che identifichi il colore.
\begin{lstlisting}
Double valued = ((double) dim / sum) * 200;
Integer value = valued.intValue();
String color = "#FF"+Integer.toHexString(value)+Integer.toHexString(value);
\end{lstlisting}
In questo modo pi\`u \`e piccolo il rapporto maggiore sar\`a l'intensit\`a del rosso.Non \`e stata sfruttata tutta la gamma di colori del modello RGB,questo perch\`e con numeri compresi tra 200 e 255 si ottengono colorazioni troppo vicine al bianco.
\\\textbf{Spazio disponibile non superato}\\Il funzionamento in questo caso \`e lo stesso, vengono solamente invertiti il rapporto e l' ordine della stringa.
\begin{lstlisting}
Double valued = ((double) sum / dim) * 200;
Integer value = valued.intValue();
String color = "#"+Integer.toHexString(value)+Integer.toHexString(value)+"FF";
\end{lstlisting}
Una volta ottenuto il colore basta andarlo a sostituire nel \textit{fill} della stanza.
Il codice usato per la colorazione delle stanze \`e localizzato in due zone differenti: quello usato dall'app Android si trova nella classe \textit{ImageHandling}(package com.auleSVGWT.server), mentre il codice relativo all Web app si trova nella classe \textit{ShowFloorPresenter}(package com.auleSVGWT.client.presenter), pronto a diventare Javascript dopo la compilazione.



\begin{thebibliography}{99}

\bibitem{hibernate} 
Christian Bauer, Gavin King;
\textit Java Persistence with Hibernate

\bibitem{gwt} 
Adam Tacy, Robert Hanson, Jason Essington, Anna Tokke;
\textit GWT in Action

\bibitem{java}
Bruce Eckel;
\textit Thinking in Java 4rd edition

\bibitem{gwtDev}
GWT Developers;
\textit http://www.gwtproject.org/doc/latest/tutorial/

\bibitem{hibDev}
Jboss;
\textit https://docs.jboss.org/hibernate/orm

\bibitem{maven}
Maven;
\textit https://maven.apache.org/guides/

\end{thebibliography}

\newpage

\chapter*{Ringraziamenti}
\end{document}


